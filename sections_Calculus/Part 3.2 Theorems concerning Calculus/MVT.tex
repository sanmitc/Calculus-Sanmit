\section{Mean Value Theorems Introduction}

Mean value theorems are very important when it comes to analyzing function behavior when we know the behavior of the function at the endpoints. 

There are generally two types of mean value theorem that we are going to study. One is the \textbf{Lagrange Mean Value theorem and the other is the Cauchy Mean Value theorem}.

\section{Lagrange's Mean Value Theorem}

\begin{outline}
    If a function f(x) is :
    \\
    a. Continuous in the closed interval of [a,b], i.e. continuous at each point in the interval [a,b], and
    \\
    b. differentiable in the open interval (a,b), i.e. differentiable at each point in the interval (a,b), 
    \\
    then there exists at least one c in the interval (a,b) such that $f'(c)=\frac{f(a)-f(b)}{a-b}$
\end{outline}


\textbf{Proof}
Consider the function:
$F(x)=Ax+f(x)$
and find a A such that the value of the function $F(x)$
makes $F(a)=F(b)$.

Such a function can be solved and it turns out that the function is:

$$F(x) = f(x) -\frac{f(b)-f(a)}{b-a}x $$

in which case we can use Rolle's theorem and recover a point within this interval so that $f'(c)=\frac{f(b)-f(a)}{b-a}$

\begin{tikzpicture}[scale=1.5]
  % Axes
  \draw[->] (-0.5,0) -- (5,0) node[right] {$x$};
  \draw[->] (0,-0.5) -- (0,3) node[above] {$f(x)$};
  
  % Function curve
  \draw[domain=0.5:4.5,smooth,variable=\x,blue] plot ({\x},{0.3*\x*\x-1.2*\x+2});
  
  % Tangent line
  \def\a{1.5} % x-coordinate of the point where f'(c) = slope of tangent
  \pgfmathsetmacro{\fa}{0.3*\a*\a-1.2*\a+2} % y-coordinate of f(a)
  \pgfmathsetmacro{\faprime}{0.6*\a-1.2} % slope of f(x) at x=a
  \pgfmathsetmacro{\tangent}{\fa+\faprime*(\x-\a)} % equation of the tangent line
  
  \draw[red,dashed] (\a-0.5,{\tangent-\faprime*0.5}) -- (\a+0.5,{\tangent+\faprime*0.5}) node[right] {$f'(c)(x-a)$};

  \draw[blue,dashed] (0.5,0.3*0.5*0.5-1.2*0.5+2) -- (2.35,0.3*2.35*2.35-1.2*2.35+2) node[right] {avg};
  
  % Points
  \filldraw (\a,\fa) circle (1pt) node[below left] {$A(a,f(a))$};
  \filldraw (\a,\tangent) circle (1pt) node[above right] {$T$};
  
  % Vertical line
  \draw[dotted] (\a,0) -- (\a,\fa) node[midway,right] {$f'(c)$};
  
  % Labels
  \node at (3,2) {$f(x)$};
  \node at (5,0.8) {$y=f(a)+(x-a)f'(c)$};
  \node at (\a,-0.2) {$a$};
\end{tikzpicture}

So, this picture is a graphical representation of MVT; here, we can see the plotted version of a function, a tangential line, and its slope at x=1.5. It is equal to the average slope of the function between 0.5 and 3

\subsection{Some Examples}

\begin{enumerate}
    \item Let $f(x)$ and $g(x)$ be differentiable functions in $(a, b)$, continuous at $a$ and $b$, and $g(x) \neq 0$ in $[a, b]$. Then prove that $\frac{g(a) f(b)-f(\dot{a}) g(b)}{g(c) f^{\prime}(c)-f(c) g^{\prime}(c)}=\frac{(b-a) g(a) g(b)}{(g(c))^2}$ for at least one $c \in(a, b)$.

Sol. We have to prove
$$
\frac{g(a) f(b)-f(a) g(b)}{g(c) f^{\prime}(c)-f(c) g^{\prime}(c)}=\frac{(b-a) g(a) g(b)}{(g(c))^2}
$$

After rearranging, we have
$$
\frac{g(c) f^{\prime}(c)-f(c) g^{\prime}(c)}{(g(c))^2}=\frac{\frac{f(b)}{g(b)}-\frac{f(a)}{g(a)}}{(b-a)}
$$

Let $h(x)=\frac{f(x)}{g(x)}$
As $f(x)$ and $g(x)$ are differentiable functions in $(a, b), h(x)$ will also be differentiable in $(a, b)$.
So, according to Lagrange's mean value theorem, there exists at least one $c \in(a, b)$ such that $h^{\prime}(c)=\frac{h(b)-h(a)}{b-a}$, which proves the required result.

\item 
Using mean value theorem, show that
$$
\frac{\beta-\alpha}{1+\beta^2}<\tan ^{-1} \beta-\tan ^{-1} \alpha<\frac{\beta-\alpha}{1+\alpha^2}, \beta>\alpha>0 .
$$

Sol. Consider function $f(x)=\tan ^{-1} x, x \in[\alpha, \beta]$.
Since $f(x)$ is continuous and differentiable, there exists at least one $c \in(\alpha, \beta)$ such that
$$
\frac{f(\beta)-f(\alpha)}{\beta-\alpha}=f^{\prime}(c)
$$

Now, $f^{\prime}(x)=\frac{1}{1+x^2}$
$$
\therefore \quad f^{\prime}(c)=\frac{1}{1+c^2}
$$

Now, $\alpha<c<\beta$
$$
\begin{array}{ll}
\Rightarrow \quad & \alpha^2<c^2<\beta^2 \\
\Rightarrow & 1+\alpha^2<1+c^2<1+\beta^2 \\
\Rightarrow & \frac{1}{1+\alpha^2}>\frac{1}{1+c^2}>\frac{1}{1+\beta^2} \\
\Rightarrow & \frac{1}{1+\beta^2}<f^{\prime}(c)<\frac{1}{1+\alpha^2} \\
\Rightarrow \quad & \frac{1}{1+\beta^2}<\frac{f(\beta)-f(\alpha)}{\beta-\alpha}<\frac{1}{1+\alpha^2} \\
\Rightarrow \quad & \frac{\beta-\alpha}{1+\beta^2}<f(\beta)-f(\alpha)<\frac{\beta-\alpha}{1+\alpha^2} \\
\Rightarrow \quad & \frac{\beta-\alpha}{1+\beta^2}<\tan ^{-1} \beta-\tan ^{-1} \alpha<\frac{\beta-\alpha}{1+\alpha^2}
\end{array}
$$
\end{enumerate}


\section{Cauchy Mean value theorem}

CMVT, or the Cauchy Mean Value theorem, is a more general version of the Lagrange Mean Value Theorem. It says that if we have two functions, f(x) and g(x), in the interval [a,b], such that they are continuous in the closed interval [a,b]. Differentiable in the open interval (a,b), then there exists a point c inside this interval such that:

$$\frac{f'(c)}{g'(c)}=\frac{f(b)-f(a)}{g(b)-g(a)}$$

\subsection{Some examples}

Let $f$ be continuous on $[a, b], a>0$, and differentiable on $(a, b)$. Prove that there exists $c \in(a, b)$ such that $\frac{b f(a)-a f(b)}{b-a}=f(c)-c f^{\prime}(c)$.

\begin{outline}
Sol. $\frac{b f(a)-a f(b)}{b-a}=f(c)-c f^{\prime}(c)$ i.e., $\frac{\frac{f(a)}{a}-\frac{f(b)}{b}}{\frac{1}{a}-\frac{1}{b}}=\frac{\frac{f(c)-c f^{\prime}(c)}{c^2}}{\frac{1}{c^2}}$

This suggests we have to consider the functions.
$$
g(x)=\frac{f(x)}{x} \text { and } h(x)=\frac{1}{x}
$$

Both the functions $g(x)$ and $h(x)$ are continuous and differentiable in $(a, b)$.
Therefore, there exists at least one $c \in(a, b)$ for which or
$$
\begin{aligned}
& \frac{g(a)-g(b)}{h(a)-h(b)}=\frac{g^{\prime}(c)}{f^{\prime}(c)} \\
& \frac{\frac{f(a)}{a}-\frac{f(b)}{b}}{\frac{1}{a}-\frac{1}{b}}=\frac{\frac{f(c)-c f^{\prime}(c)}{c^2}}{\frac{1}{c^2}}
\end{aligned}
$$
or $\quad \frac{b f^{\prime}(a)-a f(b)}{b-a}=f(c)-c f^{\prime}(c)$
Hence proved.
\end{outline}


\subsection{Exercises}

1. Find $c$ of Lagrange's mean value theorem for the function $f(x)=3 x^2+5 x+7$ in the interval $[1,3]$.\\
2. If $f(x)$ is continuous in $[a, b]$ and differentiable in $(a, b)$, then prove that there exists at least one $c \in(a, b)$ such that $\frac{f^{\prime}(c)}{3 c^2}=\frac{f(b)-f(a)}{b^3-a^3}$.\\
3. If $a, b \in R$ and $a<b$, then prove that there exists at least one real number $c \in(a, b)$ such that $\frac{b^2+a^2}{4 c^2}=\frac{c}{a+b}$.\\

4. If $f(x)$ and $g(x)$ are continuous functions in $[a, b]$ and are differentiable in $(a, b)$, then prove that there exists at least one $c \in(a, b)$ for which
$$
\left|\begin{array}{ll}
f(a) & f(b) \\
g(a) & g(b)
\end{array}\right|=(b-a)\left|\begin{array}{ll}
f(a) & f^{\prime}(c) \\
g(a) & g^{\prime}(c)
\end{array}\right| \text {, where } a<c<b .
$$\\
5. Prove that $\left|\tan ^{-1} x-\tan ^{-1} y\right| \leq|x-y| \forall x, y \in R$.\\
6. Using Lagrange's mean value theorem, prove that $\frac{b-a}{b}<\log \left(\frac{b}{a}\right)<\frac{b-a}{a}$, where $0<a<b$.\\
7. If $a>b>0$, with the aid of Lagrange's mean value theorem, prove that
(a) $n b^{n-1}(a-b)<a^n-b^n<n a^{n-1}(a-b)$, if $n>1$.
(b) $n b^{n-1}(a-b)>a^n-b^n>n a^{n-1}(a-b)$, if $0<n<1$.\\
8. Let $f(x)$ and $g(x)$ be two functions which are defined and differentiable for all $x \geq x_0$. If $f\left(x_0\right)=g\left(x_0\right)$ and $f^{\prime}(x)>g^{\prime}(x)$ for all $x>x_0$, then prove that $f(x)>g(x)$ for all $x>x_0$.\\
9. If $f(x)$ is differentiate in $[a, b]$, then prove that there exists at least one $c \in(a, b)$ such that $\left(a^2-b^2\right) f^{\prime}(c)$ $=2 c(f(a)-f(b))$.\\

