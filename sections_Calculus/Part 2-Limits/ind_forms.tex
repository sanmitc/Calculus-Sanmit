\section{Indeterminate forms}

Now that we have been well-versed in all sorts of tricks and strategies to find limits, we will focus on the specific types of indeterminate forms that arise in calculating limits and how to resolve each of their types. 


\textbf{What are indeterminant forms?}

Indeterminate forms arise in calculus when evaluating limits of functions that do not have a clearly defined value as they approach a certain point. These forms typically involve expressions like $0/0$, $\infty/\infty$, $0 \times \infty$, $\infty - \infty$, $0^0$, $1^\infty$, $\infty^0$, and $\infty^\infty$. The term "indeterminate" signifies that the limit cannot be determined simply by evaluating the function at the given point.

Here's a brief overview of each indeterminate form with examples:

\begin{enumerate}
    \item $0/0$ form: This form arises when both the numerator and the denominator of a fraction approach zero as the independent variable approaches a certain point. Examples include $\lim_{x \to 0} \left(\frac{\sin(x)}{x}\right)$ and $\lim_{x \to 1} \left(\frac{x^2 - 1}{x - 1}\right)$.
    
    \item $\infty/\infty$ form occurs when the numerator and denominator of a fraction approach infinity as the independent variable approaches a certain point. Examples include $\lim_{x \to \infty} \left(\frac{x^2}{x}\right)$.
    
    \item $0 \times \infty$ form: This form occurs when the limit of a product is not immediately obvious because one factor approaches zero while the other approaches infinity. Examples include $\lim_{x \to 0} \left(x \times \frac{1}{x}\right)$ and $\lim_{x \to \infty} \left(\frac{1}{x} \times x^2\right)$.
    
    \item $\infty - \infty$ form: This form arises when the limit of the difference between two functions tends to infinity but with an indeterminate result. Examples include $\lim_{x \to \infty} \left(x^2 - x\right)$ and $\lim_{x \to 0} \left(\frac{1}{x} - \frac{1}{x^2}\right)$.
    
    \item $0^0$ form: This form arises when a function approaches zero as the independent variable approaches a certain point, and another function approaches $0$ raised to the power of zero. Examples include $\lim_{x \to 0} \left(x^x\right)$ .
    
    \item $1^\infty$ form: This form occurs when a function approaches $1$ and another function approaches infinity as the independent variable approaches a certain point. Examples include $\lim_{x \to \infty} \left(1 + \frac{1}{x}\right)^x$ and $\lim_{x \to 0} \left(1 + x\right)^{1/x}$.
    
    \item $\infty^0$ form: This form arises when a function approaches infinity and another function approaches zero raised to the power of zero. Examples include $\lim_{x \to \infty} \left(x^{1/x}\right)$ and $\lim_{x \to \infty} \left(e^x\right)^{-1/x}$
\end{enumerate}

When dealing with indeterminate forms, additional techniques such as L'Hôpital's Rule, algebraic manipulation, or applying limit properties may be necessary to evaluate the limit. 

\textbf{Our main motivation will always be to work with 0/0 or $\infty/\infty$ forms because we can use Algebraic properties and L'hopital rule to simplify and calculate the limit. We can convert every limit to this desired format. The general prescription is given below, but often, we might deviate from these hard-coded rules and use our clarity to simplify the problems.}

Why do we choose this form 0/0 though?

\begin{enumerate}
    \item \textbf{Applicability of L'Hôpital's Rule:} Converting into "0/0" form allows direct application of L'Hôpital's Rule, simplifying limit evaluation.
    
    \item \textbf{Simplification of Algebraic Manipulation:} "0/0" form simplifies algebraic manipulation, making limit-solving more straightforward. \textbf{We can take the help of rationalization, factorization, grouping terms and leading order approximations more in this format.}
    
    \item \textbf{Clarity in Limit Evaluation:} Expressing in "0/0" form provides clearer insight into function behavior near the point of interest.
    
    \item \textbf{Consistency in Problem-solving:} Standardizing the form maintains consistency in limit-solving approaches. Otherwise, given the wide variety of functions that we encounter in real life, this whole routine is a mess. 
\end{enumerate}

\begin{table}[htbp]
\centering
\begin{tabular}{|p{1cm}|c|p{5cm}|c|}
\hline
\textbf{IDF} & \textbf{Conditions} & \textbf{Transformation to} $\boldsymbol{0/0}$ & \textbf{Transformation to} $\boldsymbol{\infty/\infty}$ \\
\hline
$0/0$ & $\lim_{x \to c} f(x) = 0,$ & $\lim_{x \to c} \frac{f(x)}{g(x)} = \lim_{x \to c} \frac{1/g(x)}{1/f(x)}$ & $\lim_{x \to c} f(x) = \infty,$ \\
& $\lim_{x \to c} g(x) = 0$ & & $\lim_{x \to c} g(x) = \infty$ \\
\hline
$\infty/\infty$ & $\lim_{x \to c} f(x) = \infty,$ & $\lim_{x \to c} \frac{f(x)}{g(x)} = \lim_{x \to c} \frac{1/g(x)}{1/f(x)}$ & $\lim_{x \to c} f(x) = \infty,$ \\
& $\lim_{x \to c} g(x) = \infty$ & & $\lim_{x \to c} g(x) = \infty$ \\
\hline
$0 \cdot \infty$ & $\lim_{x \to c} f(x) = 0,$ & $\lim_{x \to c} \frac{f(x)}{1/g(x)}$ & $\lim_{x \to c} \frac{g(x)}{1/f(x)}$ \\
& $\lim_{x \to c} g(x) = \infty$ & & \\
\hline
$\infty - \infty$ & $\lim_{x \to c} (f(x) - g(x))$ & $\lim_{x \to c} \frac{1/g(x) - 1/f(x)}{1/(f(x)g(x))}$ & $\ln \lim_{x \to c} \frac{e^{f(x)}}{e^{g(x)}}$ \\
\hline
$0^0$ & $\lim_{x \to c} f(x) = 0^+,$ & $\lim_{x \to c} f(x)^{g(x)} = \exp \lim_{x \to c} \frac{g(x)}{1/\ln f(x)}$ & $\exp \lim_{x \to c} \frac{\ln f(x)}{1/g(x)}$ \\
& $\lim_{x \to c} g(x) = 0$ & & \\
\hline
$1^\infty$ & $\lim_{x \to c} f(x) = 1,$ & $\lim_{x \to c} f(x)^{g(x)} = \exp \lim_{x \to c} \frac{\ln f(x)}{1/g(x)}$ & $\exp \lim_{x \to c} \frac{g(x)}{1/\ln f(x)}$ \\
& $\lim_{x \to c} g(x) = \infty$ & & \\
\hline
$\infty^0$ & $\lim_{x \to c} f(x) = \infty,$ & $\lim_{x \to c} f(x)^{g(x)} = \exp \lim_{x \to c} \frac{g(x)}{1/\ln f(x)}$ & $\exp \lim_{x \to c} \frac{\ln f(x)}{1/g(x)}$ \\
& $\lim_{x \to c} g(x) = 0$ & & \\
\hline
\end{tabular}
\end{table}


\begin{outline}
    So, what we have is 7 kinds of indeterminate forms and some approaches of solving the problems such as:
    \begin{enumerate}
        \item Converting the indeterminate forms to 0/0.
        \item L'hopital's rule
        \item Sandwich rule
        \item Taylor expansion rule
        \item Algebraic manipulation like rationalization, factorization, grouping, and leading order approximation.
    \end{enumerate}
\end{outline}



\section{Popular Limits using standard strategies}

We will prove the following limits using Taylor series expansion:

\begin{outline}
\begin{enumerate}
    \item $\lim_{x \to 0} \frac{\sin(x)}{x} = 1$
    \item $\lim_{x \to 0} \frac{\tan(x)}{x} = 1$
    \item $\lim_{x \to 0} \frac{e^x - 1}{x} = 1$
    \item $\lim_{x \to 0} \frac{\ln(1 + x)}{x} = 1$
\end{enumerate}
\end{outline}

\subsection*{1. $\lim_{x \to 0} \frac{\sin(x)}{x} = 1$}

The Taylor series expansion of $\sin(x)$ about $x = 0$ is:
\[
\sin(x) = x - \frac{x^3}{3!} + \frac{x^5}{5!} - \frac{x^7}{7!} + \ldots
\]
Dividing each term by $x$, we get:
\[
\frac{\sin(x)}{x} = 1 - \frac{x^2}{3!} + \frac{x^4}{5!} - \frac{x^6}{7!} + \ldots
\]
Now, as $x$ tends to $0$, all the terms with $x$ in the expansion approach $0$. Therefore, the limit of $\frac{\sin(x)}{x}$ as $x$ approaches $0$ is simply the coefficient of the $x^0$ term in the expansion, which is $1$.

\subsection*{2. $\lim_{x \to 0} \frac{\tan(x)}{x} = 1$}

The Taylor series expansion of $\tan(x)$ about $x = 0$ is:
\[
\tan(x) = x + \frac{x^3}{3} + \frac{2x^5}{15} + \frac{17x^7}{315} + \ldots
\]
Dividing each term by $x$, we get:
\[
\frac{\tan(x)}{x} = 1 + \frac{x^2}{3} + \frac{2x^4}{15} + \frac{17x^6}{315} + \ldots
\]
As $x$ tends to $0$, all the terms with $x$ in the expansion approach $0$. Therefore, the limit of $\frac{\tan(x)}{x}$ as $x$ approaches $0$ is simply the coefficient of the $x^0$ term in the expansion, which is $1$.

\subsection*{3. $\lim_{x \to 0} \frac{e^x - 1}{x} = 1$}

The Taylor series expansion of $e^x$ about $x = 0$ is:
\[
e^x = 1 + x + \frac{x^2}{2!} + \frac{x^3}{3!} + \ldots
\]
Subtracting $1$ from both sides, we get:
\[
e^x - 1 = x + \frac{x^2}{2!} + \frac{x^3}{3!} + \ldots
\]
Dividing each term by $x$, we get:
\[
\frac{e^x - 1}{x} = 1 + \frac{x}{2!} + \frac{x^2}{3!} + \ldots
\]
As $x$ tends to $0$, all the terms with $x$ in the expansion approach $0$. Therefore, the limit of $\frac{e^x - 1}{x}$ as $x$ approaches $0$ is simply the coefficient of the $x^0$ term in the expansion, which is $1$.

\subsection*{4. $\lim_{x \to 0} \frac{\ln(1 + x)}{x} = 1$}

The Taylor series expansion of $\ln(1 + x)$ about $x = 0$ is:
\[
\ln(1 + x) = x - \frac{x^2}{2} + \frac{x^3}{3} - \frac{x^4}{4} + \ldots
\]
Dividing each term by $x$, we get:
\[
\frac{\ln(1 + x)}{x} = 1 - \frac{x}{2} + \frac{x^2}{3} - \frac{x^3}{4} + \ldots
\]
As $x$ tends to $0$, all the terms with $x$ in the expansion approach $0$. Therefore, the limit of $\frac{\ln(1 + x)}{x}$ as $x$ approaches $0$ is simply the coefficient of the $x^0$ term in the expansion, which is $1$.


\section{0/0 form}

We have laid down the different strategies to solve the limits from a theoretical viewpoint. Now, we will solve some real examples and apply these strategies to calculate the limits. \textbf{0/0 form is very similar to $\infty/\infty$ form and also with $\infty-\infty$ form. We might also discuss some things related to these limits on the go.}

\subsection{Purely Algebraic Manipulation: Factorization, Grouping and Rationalization}

\begin{enumerate}
    \item $\lim_{x\to 2} \frac{x^2-5x+6}{x^2-4}$\\\\
    \textbf{Solution:}
    If we carry out a direct substitution in this case, we see that the limit becomes 0/0. Thus, we must use some algebraic manipulations to convert it to a tractable form. \textbf{We factorize the numerator and denominator and cancel the common terms.}

    \begin{align}
        & \lim_{x\to 2} \frac{x^2-5x+6}{x^2-4} \notag \\
        & =\lim_{x\to 2}  \frac{(x-2)(x-3)}{(x-2)(x+2)} \notag \\
        & =\lim_{x\to 2} \frac{x-3}{x+2} \notag \\
        & =\frac{2-3}{2+2} \notag \\
        & =-\frac{1}{4} \notag 
     \end{align}

     \item$\lim _{x \rightarrow 1}\left(\frac{2}{1-x^2}+\frac{1}{x-1}\right)$\\\\
\textbf{Solution}: We have

\begin{align}
& \lim _{x \rightarrow 1}(  \left.\frac{2}{1-x^2}+\frac{1}{x-1}\right) \notag \\
& \quad=\lim _{x \rightarrow 1}\left(\frac{2}{1-x^2}-\frac{1}{1-x}\right) \quad (\infty-\infty \text { form }) \notag 
\end{align}


When $x=1$, the expression $\frac{2}{1-x^2}-\frac{1}{1-x}$ assumes the form $\infty-\infty$. So, we need some simplification to express it in the form $\frac{0}{0}$. Then,
$$
\lim _{x \rightarrow 1}\left(\frac{2}{1-x^2}-\frac{1}{1-x}\right)=\lim _{x \rightarrow 1} \frac{2-(1+x)}{1-x^2}=\lim _{x \rightarrow 1} \frac{1-x}{1-x^2}=\lim _{x \rightarrow 1} \frac{1}{1+x}=\frac{1}{2}
$$


\item Evaluate $\lim _{x \rightarrow \frac{3 \pi}{4}} \frac{1+\sqrt[3]{\tan x}}{1-2 \cos ^2 x}$\\\\
\textbf{Solution}
$$
\begin{aligned}
& \lim _{x \rightarrow \frac{3 \pi}{4}} \frac{1+\sqrt[3]{\tan x}}{1-2 \cos ^2 x} \\
& =\lim _{x \rightarrow \frac{3 \pi}{4}} \frac{1+(\tan x)^{1 / 3}}{-\cos 2 x} \cdot \frac{1-(\tan x)^{1 / 3}+(\tan x)^{2 / 3}}{1-(\tan x)^{1 / 3}+(\tan x)^{2 / 3}} \\
& =-\lim _{x \rightarrow \frac{3 \pi}{4}} \frac{1+\tan x}{1-\tan ^2 x} \cdot \frac{(1+\tan ^2 x)}{3} \\
& =-\lim _{x \rightarrow \frac{3 \pi}{4}} \frac{1+\tan ^2 x}{1-\tan x} \cdot \frac{1}{3}=-\frac{1+1}{1-(-1)} \cdot \frac{1}{3}=-\frac{1}{3}
\end{aligned}
$$

\item \textbf{Rationalization}Evaluate $\lim _{x \rightarrow a} \frac{\sqrt{a+2 x}-\sqrt{3 x}}{\sqrt{3 a+x}-2 \sqrt{x}},(a \neq 0)$\\\\

\textbf{Solution}
We have,
$$
\begin{aligned}
& \lim _{x \rightarrow a} \frac{\sqrt{a+2 x}-\sqrt{3 x}}{\sqrt{3 a+x}-2 \sqrt{x}} \\
& =\lim _{x \rightarrow a} \frac{(\sqrt{a+2 x}-\sqrt{3 x})(\sqrt{a+2 x}+\sqrt{3 x})}{(\sqrt{3 a+x}-2 \sqrt{x})(\sqrt{3 a+x}+2 \sqrt{x})}  \times \frac{(\sqrt{3 a+x}+2 \sqrt{x})}{(\sqrt{a+2 x}+\sqrt{3 x})} \quad\left(\text { form } \frac{0}{0}\right) \\
& =\lim _{x \rightarrow a} \frac{(a+2 x-3 x)}{(3 a+x-4 x)} \frac{(\sqrt{3 a+x}+2 \sqrt{x})}{(\sqrt{a+2 x}+\sqrt{3 x})} \\
& =\lim _{x \rightarrow a} \frac{\sqrt{3 a+x}+2 \sqrt{x}}{3(\sqrt{a+2 x}+\sqrt{3 x})} \\
& =\frac{\sqrt{3 a+a}+2 \sqrt{a}}{3(\sqrt{a+2 a}+\sqrt{3 a})}=\frac{1}{3} \cdot \frac{4 \sqrt{a}}{2 \sqrt{3 a}}=\frac{2}{3 \sqrt{3}}
\end{aligned}
$$

\item 
\textbf{Rationalization}Evaluate
$$
\lim _{x \rightarrow \pi / 2} \tan ^2 x\left(\sqrt{2 \sin ^2 x+3 \sin x+4}-\sqrt{\sin ^2 x+6 \sin x+2}\right) \text {. }
$$\\\\

\textbf{Solution:} Rationalizing we get
$$
\begin{aligned}
& \lim _{x \rightarrow \pi / 2} \tan ^2 x \frac{\left(2 \sin ^2 x+3 \sin x+4-\sin ^2 x-6 \sin x-2\right)}{\sqrt{2 \sin ^2 x+3 \sin x+4}+\sqrt{\sin ^2 x+6 \sin x+2}} \\
& =\lim _{x \rightarrow \pi / 2} \frac{\sin ^2 x(\sin x-1)(\sin x-2)}{\left(1-\sin ^2 x\right)(\sqrt{9}+\sqrt{9})} \\
& =\lim _{x \rightarrow \pi / 2} \frac{-\sin ^2 x(\sin x-2)}{6(1+\sin x)} \\
& =\frac{-1(1-2)}{6(1+1)}=\frac{1}{12}
\end{aligned}$$




\item \textbf{Sandwich Theorem}
Evaluate $\lim _{n \rightarrow \infty} \frac{1}{1+n^2}+\frac{1}{2+n^2}+\cdots+\frac{n}{n+n^2}$.\\\\

\textbf{Solution}: $P_n=\frac{1}{1+n^2}+\frac{2}{2+n^2}+\cdots+\frac{n}{n+n^2}$
Now, $P_n<\frac{1}{1+n^2}+\frac{2}{1+n^2}+\cdots+\frac{n}{1+n^2}$
$$
\begin{aligned}
& =\frac{1}{1+n^2}(1+2+3+\cdots+n) \\
& =\frac{n(n+1)}{2\left(1+n^2\right)}
\end{aligned}
$$

Also,
$$
\begin{aligned}
P_n & >\frac{1}{n+n^2}+\frac{2}{n+n^2}+\frac{3}{n+n^2}+\cdots+\frac{n}{n+n^2} \\
& =\frac{n(n+1)}{2\left(n+n^2\right)}
\end{aligned}
$$

Thus, 

\begin{align}
    & \frac{n(n+1)}{2\left(n+n^2\right)}<P_n<\frac{n(n+1)}{2\left(1+n^2\right)} \notag \\
& \text{or} \quad \lim _{n \rightarrow \infty} \frac{n(n+1)}{2\left(n+n^2\right)}<\lim _{n \rightarrow \infty} P_n<\lim _{n \rightarrow \infty} \frac{n(n+1)}{2\left(1+n^2\right)} \notag \\
& \text{or} \quad \lim _{n \rightarrow \infty} \frac{1\left(1+\frac{1}{n}\right)}{2\left(\frac{1}{n}+1\right)}<\lim _{n \rightarrow \infty} P_n<\lim _{n \rightarrow \infty} \frac{1\left(1+\frac{1}{n}\right)}{2\left(\frac{1}{n^2}+1\right)} \notag \\
& \text{or} \quad \frac{1}{2}<\lim _{n \rightarrow \infty} P_n<\frac{1}{2} \notag \\
& \implies \quad \lim _{n \rightarrow \infty} P_n=\frac{1}{2} \notag 
\end{align}

\item Evaluate $\lim _{x \rightarrow 0} \frac{\sqrt[3]{8+x}-\sqrt[3]{8+x^2-x^3}}{\sqrt[3]{8+x}-\sqrt[3]{8+x^2+x^3}}$\\\\

We know, $(a-b)\left(a^2+a b+b^2\right)=a^3-b^3$.
$\therefore$ limit
$$
\begin{aligned}
& =\lim _{x \rightarrow 0} \frac{x-x^2+x^3}{x-x^2-x^3} \\
& \times \frac{(\sqrt[3]{8+x})^2+\sqrt[3]{8+x} \cdot \sqrt[3]{8+x^2+x^3}+\left(\sqrt[3]{8+x^2+x^3}\right)^2}{(\sqrt[3]{8+x})^2+\sqrt[3]{8+x} \cdot \sqrt[3]{8+x^2-x^3}+\left(\sqrt[3]{8+x^2-x^3}\right)^2} \\
& =\lim _{x \rightarrow 0} \frac{1-x+x^2}{1-x-x^2} \\
& \times \frac{(\sqrt[3]{8+x})^2+\sqrt[3]{8+x} \cdot \sqrt[3]{8+x^2+x^3}+\left(\sqrt[3]{8+x^2+x^3}\right)^2}{(\sqrt[3]{8+x})^2+\sqrt[3]{8+x} \cdot \sqrt[3]{8+x^2-x^3}+\left(\sqrt[3]{8+x^2-x^3}\right)^2} \\
& =\frac{1}{1} \cdot \frac{2^2+2 \cdot 2+2^2}{2^2+2 \cdot 2+2^2}=1 . \\
&
\end{aligned}
$$
\newpage
\item 3. Evaluate
$$
\lim _{n \rightarrow \infty} \frac{1 \cdot \sum_1^n r+2 \cdot \sum_1^{n-1} r+3 \cdot \sum_1^{n-2} r+\ldots+n \cdot 1}{n^4}
$$\\\\

Consider
$$
\begin{aligned}
& m \cdot \sum_1^{n-m+1} r \\
& =m\{1+2+3+\ldots+\text { to }(n-m+1) \text { terms }\} \\
& =m \cdot \frac{(n-m+1)(n-m+2)}{2} \\
& =\frac{m}{2}\left\{n^2-(2 m-3) n+(m-1)(m-2)\right\} \\
& =\frac{n^2}{2} \cdot m-\frac{n}{2} m(2 m-3)+\frac{m}{2}\left(m^2-3 m+2\right) \\
& =\frac{n^2}{2} \cdot m-n \cdot m^2+\frac{3 n}{2} \cdot m+\frac{m^3}{2}-\frac{3}{2} m^2+m \\
& =\left(\frac{n^2}{2}+\frac{3 n}{2}+1\right) m-\left(n+\frac{3}{2}\right) m^2+\frac{1}{2} m^3 \\
& \therefore \quad \sum_1^n\left\{m \sum_1^{n-m+1} r\right\} \\
& =\left(\frac{n^2}{2}+\frac{3 n}{2}+1\right) \sum_1^n m-\left(n+\frac{3}{2}\right) \sum_1^n m^2+\frac{1}{2} \sum_1^n m^3 \\
& =\frac{n^2+3 n+2}{2} \cdot \frac{n(n+1)}{2} -\frac{2 n+3}{2} \cdot \frac{n(n+1)(2 n+1)}{6}+\frac{1}{2} \cdot \frac{n^2(n+1)^2}{4} \\
& =\frac{n(n+1)^2(n+2)}{4}-\frac{n(n+1)(2 n+1)(2 n+3)}{12} +\frac{n^2(n+1)^2}{8} \\
& \text{Thus the Limit becomes}\\
& =\lim _{n \rightarrow \infty} \frac{1}{n^4} \cdot n^4\left\{\frac{\left(\frac{1}{n}+1\right)^2\left(\frac{2}{n}+1\right)}{4}\right. \left.\quad-\frac{\left(1+\frac{1}{n}\right)\left(2+\frac{1}{n}\right)\left(2+\frac{3}{n}\right)}{12}+\frac{\left(1+\frac{1}{n}\right)^2}{8}\right\} \\
& =\frac{1}{4}-\frac{4}{12}+\frac{1}{8} \\
& =\frac{1}{4}-\frac{1}{3}+\frac{1}{8}=\frac{1}{24} .
\end{aligned}
$$
\end{enumerate}


\subsubsection{Short Preliminary Exercises}
\begin{enumerate}
\item $\lim_{x \to 0} \frac{x^2 \sin(\pi x)}{x - \tan(x)}$
\item $\lim_{x \to 1} \frac{\ln(1 - x^2)}{\sin(x - 1)}$
\item $\lim_{x \to 0} \frac{e^{2x} - 1 - 2x}{x^3}$
\item $\lim_{x \to 0} \frac{\cos(x^2) - 1}{\sin^2(x)}$
\item $\lim_{x \to \pi/4} \frac{\tan(x) - 1}{x - \pi/4}$
\item $\lim_{x \to 0} \frac{\sqrt{x + 1} - 1}{\sqrt{x + 4} - 2}$
\item $\lim_{x \to 2} \frac{x^2 - 4x + 4}{x^3 - 8}$
\item $\lim_{x \to -1} \frac{\sin(x + 1)}{x^2 + 2x + 1}$
\item $\lim_{x \to 0} \frac{x^4 \sin(x)}{1 - \cos(x)}$
\item $\lim_{x \to \infty} \frac{(x + 1)^2 \sin(x)}{x^3 - x^2}$
\end{enumerate}


\subsection{Using Trigonometric Identities}

We have already encountered the Trigonometric identities with proof based on the Taylor series expansion. These are :

\begin{outline}
    $$\lim_{x\to 0}\frac{\sin x}{x}=1$$ $$\lim_{x\to 0} \frac{\tan x}{x}=1$$
\end{outline}

Now, this is open to substitution, manipulation, and recombination. For example, due to these identities, the following identities also hold. 
replacing x with x-a to deal with limits of x going to a, and all sorts of things are possible. The more general version of these formulae are:

\begin{outline}
    $$\lim_{x\to b}\frac{\sin f(x)}{f(x)}=1$$ $$\lim_{x\to b} \frac{\tan f(x)}{f(x)}=1$$
    If $\lim_{x\to b} f(x)=0$
\end{outline}

\textbf{Now we use this relation to solve more limit problems}

\begin{enumerate}
    \item Evaluate $\lim _{x \rightarrow \frac{\pi}{6}} \frac{2-\sqrt{3} \cos x-\sin x}{(6 x-\pi)^2}$\\\\
\textbf{Solution} $\lim _{x \rightarrow \frac{\pi}{6}} \frac{2-\sqrt{3} \cos x-\sin x}{(6 x-\pi)^2} \quad\left(\frac{0}{0}\right.$ form $)$
$$
\begin{aligned}
& =\lim _{x \rightarrow \frac{\pi}{6}} 2 \times \frac{1-\left(\frac{\sqrt{3}}{2} \cos x+\frac{1}{2} \sin x\right)}{(6 x-\pi)^2} \\
& =\lim _{x \rightarrow \frac{\pi}{6}} 2 \times \frac{1-\cos \left(x-\frac{\pi}{6}\right)}{36\left(x-\frac{\pi}{6}\right)^2} \\
& =\lim _{x \rightarrow \frac{\pi}{6}} \frac{2 \sin ^2\left(\frac{x}{2}-\frac{\pi}{12}\right)}{18 \times 4\left(\frac{x}{2}-\frac{\pi}{12}\right)^2} \\
& =\frac{1}{36}\left(\lim _{x \rightarrow \frac{\pi}{6}} \frac{\sin \left(\frac{x}{2}-\frac{\pi}{12}\right)}{\left(\frac{x}{2}-\frac{\pi}{12}\right)}\right)^2=\frac{1}{36}
\end{aligned}
$$

\newpage
\item Evaluate $\lim _{x \rightarrow-\infty}\left[\frac{x^4 \sin \left(\frac{1}{x}\right)+x^2}{\left(1+|x|^3\right)}\right]$\\\\

\textbf{Solution}
$$
\begin{gathered}
\lim _{x \rightarrow-\infty}\left[\frac{x^4 \sin \left(\frac{1}{x}\right)+x^2}{\left(1-x^3\right)}\right] \\
=\lim _{x \rightarrow-\infty}\left[\frac{x \sin \left(\frac{1}{x}\right)+\frac{1}{x}}{\frac{1}{x^3}-1}\right] \\
=\lim _{x \rightarrow-\infty}\left[\frac{\frac{\sin \left(\frac{1}{x}\right)}{\frac{1}{x}}+\frac{1}{x}}{\frac{1}{x^3}-1}\right] \\
=\frac{1+0}{0-1}=-1
\end{gathered}
$$

\item Evaluate $\lim _{x \rightarrow-1^{+}} \frac{\sqrt{\pi}-\sqrt{\cos ^{-1} x}}{\sqrt{1+x}}$.\\\\

\textbf{Solution}
$$
\begin{aligned}
\lim _{x \rightarrow-1^{+}} & \frac{\sqrt{\pi}-\sqrt{\cos ^{-1} x}}{\sqrt{1+x}} \\
& =\lim _{x \rightarrow-1^{+}} \frac{\pi-\cos ^{-1} x}{\sqrt{1+x}} \cdot \frac{1}{\sqrt{\pi}+\sqrt{\cos ^{-1} x}} \\
& =\lim _{x \rightarrow-1^{+}} \frac{\cos ^{-1}(-x)}{\sqrt{1+x}} \cdot \frac{1}{\sqrt{\pi}+\sqrt{\pi}} \\
& =\frac{1}{2 \sqrt{\pi}} \lim _{\theta \rightarrow 0^{+}} \frac{\theta}{\sqrt{1-\cos \theta}} \quad \text { (Putting } \cos ^{-1}(-x)=\theta ) \\
& =\frac{1}{2 \sqrt{\pi}} \lim _{\theta \rightarrow 0^{+}} \frac{\theta}{\sqrt{2 \sin ^2 \frac{\theta}{2}}} \\
& =\frac{1}{\sqrt{2 \pi}} \lim _{\theta \rightarrow 0^{+}} \frac{\frac{\theta}{2}}{\sin \frac{\theta}{2}} \\
& =\frac{1}{\sqrt{2 \pi}}
\end{aligned}
$$
\end{enumerate}


\subsubsection{Exercises}

\begin{enumerate}
    \item $\lim_{x\to 0} \frac{\tan x - \sin x}{x^3}$
    \item $\lim_{x\to\pi} \frac{\sin^{-1}(1+\cos x)-\sec (\frac{x}{2})}{(x-\pi)}$
    \item $\lim_{x\to \infty} x (\tan^{-1}(\frac{x+1}{x+4})-\frac{\pi}{4})$
    \item $\lim_{n \to \infty} n\sin(2\pi \sqrt{1+n^2}), n\in \mathbb{N}$
    \item $\lim_{x\to \infty} x \left[ \tan^{-1}(\frac{x+1}{x+2})- \tan^{-1}(\frac{x}{x+2})\right]$
    \item Evaluate $\lim _{h \rightarrow 0} \frac{2\left[\sqrt{3} \sin \left(\frac{\pi}{6}+h\right)-\cos \left(\frac{\pi}{6}+h\right)\right]}{\sqrt{3} h(\sqrt{3} \cos h-\sin h)}$.
    \item Evaluate $\lim _{x \rightarrow 0} \frac{8}{x^8}\left\{1-\cos \frac{x^2}{2}-\cos \frac{x^2}{4}+\cos \frac{x^2}{2} \cos \frac{x^2}{4}\right\}$
    \item Evaluate $\lim _{x \rightarrow 1}(1-x) \tan \frac{\pi x}{2}$.
    \item  Evaluate $\lim _{x \rightarrow 0} \frac{x \tan 2 x-2 x \tan x}{(1-\cos 2 x)^2}$.
\end{enumerate}


\subsection{Exponential and Logarithmic Limits}

We now use two more limit identities:

\begin{outline}
    $$\lim_{x \to 0}\frac{e^x-1}{1}=1$$
    $$\lim_{x \to 0}\frac{ln(1+x)}{x}=1$$
\end{outline}

We now solve some problems to sharpen our understanding.

\begin{itemize}
    \item Evaluate $\lim _{x \rightarrow 0} \frac{10^x-2^x-5^x+1}{x \tan x}$\\\\
\textbf{Solution}:
We have
$$
\begin{aligned}
\lim _{x \rightarrow 0} & \frac{10^x-2^x-5^x+1}{x \tan x} \\
& =\lim _{x \rightarrow 0} \frac{5^x \cdot 2^x-2^x-5^x+1}{x \tan x} \\
& =\lim _{x \rightarrow 0} \frac{\left(5^x-1\right)\left(2^x-1\right)}{x \tan x} \\
& =\lim _{x \rightarrow 0} \frac{5^x-1}{x} \frac{2^x-1}{x} \frac{x}{\tan x} \\
& =\lim _{x \rightarrow 0} \frac{5^x-1}{x} \lim _{x \rightarrow 0} \frac{2^x-1}{x} \lim _{x \rightarrow 0} \frac{x}{\tan x} \\
& =(\log 5)(\log 2)(1) \\
& =(\log 5)(\log 2)
\end{aligned}
$$


\item  Evaluate $\lim _{x \rightarrow 0} \frac{\log (5+x)-\log (5-x)}{x}$\\\\
\textbf{Solution}: We have $\lim _{x \rightarrow 0} \frac{\log (5+x)-\log (5-x)}{x} \quad\left(\frac{0}{0}\right)$ form
$$
\begin{aligned}
& =\lim _{x \rightarrow 0} \frac{\log \left\{5\left(1+\frac{x}{5}\right)\right\}-\log \left\{5\left(1-\frac{x}{5}\right)\right\}}{x} \\
& =\lim _{x \rightarrow 0} \frac{\left\{\log 5+\log \left(1+\frac{x}{5}\right)\right\}-\left\{\log 5+\log \left(1-\frac{x}{5}\right)\right\}}{x} \\
& =\lim _{x \rightarrow 0} \frac{\log \left(1+\frac{x}{5}\right)-\log \left(1-\frac{x}{5}\right)}{x} \\
& =\lim _{x \rightarrow 0} \frac{1}{5} \frac{\log \left(1+\frac{x}{5}\right)}{x / 5}+\lim _{x \rightarrow 0} \frac{\log \left(1-\frac{x}{5}\right)}{-x / 5} \frac{1}{5}=\frac{1}{5}+\frac{1}{5}=\frac{2}{5}
\end{aligned}
$$
\end{itemize}


\subsubsection{Exercises}

\begin{itemize}
    \item  $\lim _{x \rightarrow \infty}\left[x\left(a^{1 / x}-1\right)\right], a>1$
    \item  $\lim _{x \rightarrow 0} \frac{x 2^x-x}{1-\cos x}$
\item   $\lim _{x \rightarrow 2} \frac{\sin \left(e^{x-2}-1\right)}{\log (x-1)}$
\item   $\lim _{x \rightarrow 0} \frac{e^{x^2}-\cos x}{x^2}$
\item   $\lim _{x \rightarrow 0} \frac{e^x+e^{-x}-2}{x^2}$
\item   $\lim _{x \rightarrow a} \frac{\log (x-a)}{\log \left(e^x-e^a\right)}$
\item   $\lim _{x \rightarrow 0} \frac{a^{\tan x}-a^{\sin x}}{\tan x-\sin x}, a>0$
\item   $\lim _{x \rightarrow 0} \frac{\left(1-3^x-4^x+12^x\right)}{\sqrt{(2 \cos x+7)}-3}$
\end{itemize}


\subsection{Taylor Series Expansion}

Taylor series expansion is a powerful tool for the calculation of limits. It converts complicated functions into polynomial expressions that are more easily tractable and solvable using Algebraic manipulations. Taylor series expansion is significant in the real world, as it is used as a routine for approximating and expanding quantities into separable polynomials. Each term often contains special properties.


\textbf{Let us see some examples:}

\begin{enumerate}
    \item Evaluate $\lim _{x \rightarrow 0} \frac{e^{\sin x}-(1+\sin x)}{(\tan (\sin x))^2}$\\\\

\textbf{Solution}
$$
\begin{aligned}
& \lim _{x \rightarrow 0} \frac{e^{\sin x}-(1+\sin x)}{(\tan (\sin x))^2} \\
& \left.\quad=\lim _{h \rightarrow 0} \frac{e^h-(1+h)}{(\tan h)^2} \quad \text { (where } \sin x=h\right) \\
& \quad=\lim _{h \rightarrow 0} \frac{\left(1+h+\frac{h^2}{2 !}+\ldots\right)-(1+h)}{\left(h+\frac{h^3}{3}+\ldots\right)^2} \\
& \quad=\lim _{h \rightarrow 0} \frac{\frac{h^2}{2 !}}{h^2} \\
& =\frac{1}{2}
\end{aligned}
$$
\item Evaluate $\lim _{x \rightarrow 0} \frac{e^{x^2}-\cos x}{\log (1+x)-\sin x}$\\\\



\textbf{Solution}

We already have:\\\\
\begin{aligned}
e^{x^2}=1+x^2+O\left(x^4\right) \\
\cos x=1-\frac{1}{2} x^2+O\left(x^4\right)
\end{aligned}

\begin{aligned}
& \text{The numerator becomes}\\
&e^{x^2}-\cos x=\frac{3}{2} x^2+O\left(x^4\right)\\
&\begin{aligned}
\lim _{x \rightarrow 0} \frac{e^{x^2}-\cos x}{\log (1+x)-\sin x} & =\lim _{x \rightarrow 0} \frac{\frac{3}{2} x^2+O\left(x^4\right)}{-\frac{x^2}{2}+O\left(|x|^3\right)} \\
& =\lim _{x \rightarrow 0} \frac{\frac{3}{2}+O\left(x^2\right)}{-\frac{1}{2}+O(|x|)} \\
& =\frac{\frac{3}{2}}{-\frac{1}{2}}=-3
\end{aligned}
\end{aligned}

\newpage

\item Evaluate
$$
\lim _{x \rightarrow 0} \frac{\cos x-1+\frac{1}{2} x \sin x}{[\ln (1+x)]^4}
$$\\\\

\textbf{Solution}
$$
\begin{aligned}
\lim _{x \rightarrow 0} \frac{\cos x-1+\frac{1}{2} x \sin x}{[\ln (1+x)]^4} & =\lim _{x \rightarrow 0} \frac{\left[1-\frac{1}{2} x^2+\frac{1}{4 !} x^4+O\left(x^6\right)\right]-1+\frac{1}{2} x\left[x-\frac{1}{3 !} x^3+O\left(|x|^5\right)\right]}{\left[x+O\left(x^2\right)\right]^4} \\
& =\lim _{x \rightarrow 0} \frac{\left(\frac{1}{4 !}-\frac{1}{2 \times 3 !}\right) x^4+O\left(x^6\right)+\frac{x}{2} O\left(|x|^5\right)}{\left[x+O\left(x^2\right)\right]^4} \\
& =\lim _{x \rightarrow 0} \frac{\left(\frac{1}{4 !}-\frac{1}{2 \times 3 !}\right) x^4+O\left(x^6\right)+O\left(x^6\right)}{\left[x+O\left(x^2\right)\right]^4} \\
& =\lim _{x \rightarrow 0} \frac{\left(\frac{1}{4 !}-\frac{1}{2 \times 3 !}\right) x^4+O\left(x^6\right)}{[x+x O(|x|)]^4} \\
& =\lim _{x \rightarrow 0} \frac{\left(\frac{1}{4 !}-\frac{1}{2 \times 3 !}\right) x^4+x^4 O\left(x^2\right)}{x^4[1+O(|x|)]^4}  \\
& =\lim _{x \rightarrow 0} \frac{\left(\frac{1}{4 !}-\frac{1}{2 \times 3 !}\right)+O\left(x^2\right)}{[1+O(|x|)]^4} \\
& =\frac{1}{4 !}-\frac{1}{2 \times 3 !}  \\
& =\frac{1}{3 !}\left(\frac{1}{4}-\frac{1}{2}\right)=-\frac{1}{4 !} \quad
\end{aligned}
$$
\end{enumerate}


\subsubsection{Exercises}

1. Evaluate $\lim _{x \rightarrow 0}\left\{\frac{\sin x-x+\frac{x^3}{6}}{x^5}\right\}$.\\
2. Evaluate $\lim _{x \rightarrow 0} \frac{e^x-1-x}{x^2}$.\\
3. Evaluate $\lim _{x \rightarrow 0} \frac{e^x-e^{-x}-2 x}{x-\sin x}$\\.
4. If $\lim _{x \rightarrow 0} \frac{1-\cos x}{e^{a x}-b x-1}=1$ then find the values of $a$ and $b$.\\
5. Find the values of $a$ and $b$ in order that
$$
\lim _{x \rightarrow 0} \frac{x(1+a \cos x)-b \sin x}{x^3}=1 .
$$ \\

6. The integral value for which the expression:
$$\lim_{x\to 0} \frac{\cos^2x-\cos x - e^x \cos x + e^x -\frac{x^3}{2}}{x^n}$$
is a well-defined non-zero number.\\

7. $\lim _{x \rightarrow 0} \frac{x-\sin x}{e^x-1-x-\frac{x^2}{2}}$ \\
8. $\lim _{x \rightarrow 0} \frac{\ln ^2(1+x)-\sin ^2 x}{1-e^{-x^2}}$\\
9. $\lim _{x \rightarrow 0} \frac{2(\tan x-\sin x)-x^3}{x^5}$. \\

10. $\lim _{x \rightarrow 0}\left[x-x^2 \ln \left(1+\frac{1}{x}\right)\right]$.\\
11. $\lim _{x \rightarrow 0}\left(\frac{1}{x^2}-\frac{\cot x}{x}\right)$
\\
12.$\lim _{x \rightarrow 0}\left(\frac{1}{x^2}-\cot ^2 x\right)$\\

13. $\lim_{x \to 0}{\frac{x^2-\frac{x^6}{2}-x^2 \cos (x^2)}{\sin (x^{10})}}
$\\


\section{Indeterminant form: $1^\infty$}

\textbf{How to Calculate the $1^\infty$ limit?}
Let $\quad L=\lim _{x \rightarrow a} f(x)^{g(x)}$, where $\lim _{x \rightarrow a} f(x)=1$ and $\lim _{x \rightarrow a} g(x)=\infty$
$$
\begin{aligned}
\therefore \quad L & =\lim _{x \rightarrow a} f(x)^{g(x)}=\lim _{x \rightarrow a}(1+(f(x)-1))^{\frac{1}{f(x)-1 /(x)-1) \times x|x|}} \\
& =\left[\lim _{x \rightarrow a}\left((1+(f(x)-1))^{\frac{1}{f(x)-1}}\right)\right]^{\lim (f(x)-1) \times g(x)} \\
\therefore \quad L & =e^{\lim _{x \rightarrow a}(f(x)-1) \times g(x)}
\end{aligned}
$$

\begin{enumerate}
    \item Evaluate $\lim _{x \rightarrow 0}\left(\frac{\sin x}{x}\right)^{\left(\frac{\sin x}{x-\sin x}\right)}$.\\\\
\textbf{Solution}
Since $\lim _{x \rightarrow 0} \frac{\sin x}{x}=1$
and
$$
\lim _{x \rightarrow 0} \frac{\sin x}{x-\sin x}=\lim _{x \rightarrow 0} \frac{1}{\left(\frac{x}{\sin x}-1\right)}=\frac{1}{1-1}=\infty
$$

We have $\lim _{x \rightarrow 0}\left(\frac{\sin x}{x}\right)^{\left(\frac{\sin x}{x-\sin x}\right)}=e^{\lim _{x \rightarrow 0}\left(\frac{\sin x}{x}-1\right)\left(\frac{\sin x}{(x-\sin x)}\right)}$
$$
=e^{\lim _{x \rightarrow 0}-\frac{\sin x}{x}}=e^{-1}=\frac{1}{e}
$$

\item Evaluate $\lim _{x \rightarrow 0}\left(\frac{a^x+b^x+c^x}{3}\right)^{2 / x} ;(a, b, c>0)$\\\\

\textbf{Solution}.
We have:
$$
\begin{aligned}
\lim _{x \rightarrow 0}\left(\frac{a^x+b^x+c^x}{3}\right)^{2 / x} & =e^{\lim _{x \rightarrow 0}\left(\frac{a^x+b^x+c^x}{3}-1\right) \frac{2}{x}} \\
& =e^{\frac{2}{3} \lim _{x \rightarrow 0}\left(\frac{a^x+b^x+c^x-3}{x}\right)} \\
& =e^{\frac{2}{3} (\lim _{x \rightarrow 0}\left(\frac{a^x-1}{x}+\frac{b^x-1}{x}+\frac{c^x-1}{x})\right)}\\
& =e^{\frac{2}{3}\left\{\lim _{x \rightarrow 0} \frac{a^x-1}{x}+\lim _{x \rightarrow 0} \frac{b^x-1}{x}+\lim _{x \rightarrow 0} \frac{c^x-1}{x}\right\}} \\
& =e^{(2 / 3)\{\ln a+\ln b+\ln c\}} \\
& =e^{(2 / 3) \ln (a b c)} \\
& =e^{\ln (a b c)^{2 / 3}} \\
& =(a b c)^{2 / 3}
\end{aligned}
$$
\item 
If $f(n)=\lim _{x \rightarrow 0}\left\{\left(1+\sin \frac{x}{2}\right)\left(1+\sin \frac{x}{2^2}\right) \ldots\left(1+\sin \frac{x}{2^n}\right)\right\}^{\frac{1}{x}}$ then find $\lim _{n \rightarrow \infty} f(n)$.\\\\

\textbf{Solution}
$$
\begin{aligned}
& f(n)=\lim _{x \rightarrow 0} e^{\frac{1}{x}\left\{\left(1+\sin \frac{x}{2}\right)\left(1+\sin \frac{x}{2^2}\right) \ldots\left(1+\sin \frac{x}{2^n}\right)-1\right\}} \\
& =\lim _{x \rightarrow 0} e \\
& \frac{\left\{1+\left(\sin \frac{x}{2}+\sin \frac{x}{2^2}+\cdots+\sin \frac{x}{2^n}\right)+\left(\sin \frac{x}{2} \sin \frac{x}{2^2}+\cdots\right)+\cdots-1\right\}}{x} \\
& =\lim _{x \rightarrow 0} e^{\left\{\frac{\sin \frac{x}{2}}{x}+\frac{\sin \left(\frac{x}{2^2}\right)}{x}+\cdots+ \frac{\sin\left(\frac{x}{2^n}\right)}{x}\right\}} \\
& =e^{\left(\frac{1}{2}+\frac{1}{2^2}+\cdots+\frac{1}{2^n}\right)} \\
& \therefore \quad \lim _{n \rightarrow \infty} f(n)=e^{\frac{1 / 2}{1-\frac{1}{2}}}=e \\
&
\end{aligned}
$$
\item Evaluate: $\lim _{x \rightarrow \infty}\left(\frac{3 x^2+1}{4 x^2-1}\right)^{\frac{x^3}{1+x}}$.
$$
\begin{aligned}
\text { Limit } & =\lim _{x \rightarrow \infty}\left(\frac{3}{4} \cdot \frac{x^2+\frac{1}{3}}{x^2-\frac{1}{4}}\right)^{\frac{x^3}{1+x}} \\
& =\lim _{x \rightarrow \infty}\left(\frac{3}{4}\right)^{\frac{x^3}{1+x}} \cdot\left(\frac{x^2+\frac{1}{3}}{x^2-\frac{1}{4}}\right)^{\frac{x^3}{1+x}} \\
& =\lim _{x \rightarrow \infty}\left(\frac{3}{4}\right)^{\frac{x^3}{1+x}} \cdot \lim _{x \rightarrow \infty}\left(1+\frac{\frac{7}{12}}{x^2-\frac{1}{4}}\right)^{\frac{x^3}{1+x}} \\
& =\left(\frac{3}{4}\right)^{\lim _{x \rightarrow \infty} \frac{x^3}{1+x} } \cdot \lim _{x \rightarrow \infty}\left(1+\frac{\frac{7}{12}}{x^2-\frac{1}{4}}\right)^{\frac{x^3}{1+1}}
\end{aligned}
$$

Now, $\lim _{x \rightarrow \infty} \frac{x^3}{1+x}=\lim _{x \rightarrow \infty} \frac{x^2}{\frac{1}{x}+1}=\infty$,
$$
\begin{aligned}
& \therefore \quad\left(\frac{3}{4}\right)^{\lim _{x \rightarrow \infty} \frac{x^3}{1+x}}=0 \quad\left(\because 0<\frac{3}{4}<1\right) \\
& \text { and, } \lim _{x \rightarrow \infty}\left(1+\frac{\frac{7}{12}}{x^2-\frac{1}{4}}\right)^{\frac{x^3}{1+x}} \\
& =\lim _{x \rightarrow \infty}\left\{\left(1+\frac{\frac{7}{12}}{x^2-\frac{1}{4}}\right)^{(x^2-\frac{1}{4})}\right\}^{\frac{x^3}{\left(x^2-\frac{1}{4}\right)(1+x)}} \\
& =\left|e^{7 / 12}\right|^{\left(\lim _{x \rightarrow \infty} \frac{1}{\left(1-\frac{1}{4 x^2}\right)\left(\frac{1}{x}+1\right)}\right)}=\left(e^{\frac{7}{12}}\right)^1=e^{\frac{7}{12}} \\
&
\end{aligned}
$$
because $\lim _{x \rightarrow \infty}\left(1+\frac{a}{f(x)}\right)^{f(x)}=e^a$, when $f(x) \rightarrow \infty$ as $x \rightarrow \infty$,

Hence, the required limit $=0 \cdot e^{7 / 12}=0$.

\newpage
\item Evaluate: $\lim _{x \rightarrow 0}\left(\frac{\tan x}{x}\right)^{\frac{1}{x}}$.\\\\

\textbf{Solution}
Here, the form is $1^{\infty}$.
$$
\begin{aligned}
\text { Limit } & =e^{\lim _{x \rightarrow 0} \log \left(\frac{\tan x}{x}\right)^{\frac{1}{x}}} \\
& =e^{\lim _{x \rightarrow 0} \frac{1}{x} \log \left(\frac{\tan x}{x}\right)}
\end{aligned}
$$
$$ \text{Now, we start to treat the exponent}
=\lim _{x \rightarrow 0} \frac{\frac{x}{\tan x} \cdot \frac{x \sec ^2 x-\tan x}{x^2}}{1}
$$
(using L' Hospital's rule)
$$
\begin{aligned}
& =\lim _{x \rightarrow 0} \frac{x \sec ^2 x-\tan x}{x \tan x} \\
& =\lim _{x \rightarrow 0} \frac{\sec ^2 x+x \cdot 2 \sec ^2 x \tan x-\sec ^2 x}{\tan x+x \sec ^2 x}
\end{aligned}
$$
$$
\text { (form: } \frac{0}{0} \text { ) }
$$
(using L'Hospital's rule)
$$
\begin{aligned}
& =\lim _{x \rightarrow 0} \frac{2 x \sec ^2 x \cdot \tan x}{\tan x+x \sec ^2 x} \\
& =\lim _{x \rightarrow 0} \frac{2 \sec ^2 x \cdot \tan x+2 x\left(\sec ^4 x+2 \sec ^2 x \cdot \tan ^2 x\right)}{2 \sec ^2 x+x \cdot 2 \sec ^2 x \tan x} \\
& =\frac{0}{2}=0
\end{aligned}
$$
$\therefore$ from (1), limit $=e^0=1$.
\end{enumerate}


\subsubsection{Exercises}

Find the limits:

\begin{enumerate}
    \item  $\lim _{x \rightarrow \infty}(1+4 / x)^{x+3}$;
\item $\lim _{x \rightarrow 0} \frac{e^{-x}-1}{x}$;
\item $\lim _{x \rightarrow 0} \frac{a^{2 x}-1}{x}$;
\item $\lim _{x \rightarrow 0}\left(1+3 \tan ^2 x\right)^{\cot ^2 x}$;
\item $\lim _{x \rightarrow \pi / 4}(\sin 2 x)^{\tan ^2 2 x}$;
\item $\lim _{x \rightarrow \infty}\left(\frac{2 x-1}{2 x+1}\right)^x$;
\item $\lim _{x \rightarrow \pi / 2}(\tan x)^{\tan 2 x}$;
\item $\lim _{x \rightarrow \pi / 2}(\sin x)^{\tan x}$;
\item $\lim _{x \rightarrow \infty}\left(\frac{3 x^2+2 x+1}{x^2+x+2}\right)^{(6 x+1) /(3 x+2)}$;
\item $\lim _{x \rightarrow \infty}\left(\frac{1+3 x}{2+3 x}\right)^{(1-\sqrt{x}) /(1-x)}$;
\item $\lim _{x \rightarrow 0} \frac{e^{\alpha x}-e^{\beta x}}{x}$
\item $\lim _{x \rightarrow \infty}\left(\frac{x^2-2 x+2}{x^2-4 x+1}\right)^x$
\item $\lim _{x \rightarrow 0}\left\{\tan \left(\frac{\pi}{4}+x\right)\right\}^{\frac{1}{x}}$
\item $\lim (\sin x)^{\tan x}$ $x \rightarrow \frac{\pi}{2}$
\item $\lim _{x \rightarrow 0}\left(\frac{\sin x}{x}\right)^{\frac{1}{x}}$
\item $\lim _{x \rightarrow 0}(\sin x+\cos x)^{1 / x}$
\item $\lim _{x \rightarrow 0}(\sec \sqrt{  x})^{10 / x}$
\end{enumerate}


\section{Indeterminate forms $0^\infty$, $\infty^0$}

In these sorts of cases, we assume the limit to be something and then take the log on both sides to calculate the limit.

For example:

Evaluate $\lim _{x \rightarrow \frac{\pi^{-}}{2}}(\cos x)^{\cos x}$
Sol. Let $y=\lim _{\pi^{-}}(\cos x)^{\cos x} \quad\left(0^0\right.$ form)
$$

\begin{aligned}
\therefore \quad \log y & =\log \lim _{x \rightarrow \frac{\pi^{-}}{2}}(\cos x)^{\cos x} \\
& =\lim _{x \rightarrow \frac{\pi}{2}^{-}} \log \left[(\cos x)^{\cos x}\right] \\
& =\lim _{x \rightarrow \frac{\pi^{-}}{2}} \frac{\log (\cos x)}{\sec x} \\
& =\lim _{x \rightarrow \frac{\pi^{-}}{2}} \frac{-\sin x}{\cos x} \\
& =\lim _{x \rightarrow \frac{\pi^{-}}{2}}(-\cos x)=0 \\
\therefore \quad y=1 & \text { (Using L'Hospital's rule) }
\end{aligned}
$$



