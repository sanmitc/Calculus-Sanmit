Let us go back to the chapter on limits and recall the example of instantaneous velocity. We saw that although we can measure the average velocity just by seeing the initial and the final displacement, divided by the total time, this only captured part of the picture. To capture the whole scenario, we had to shrink the time interval we were considering. We had to see what was happening in a time instant and define the instantaneous velocity similarly.

Now, the instantaneous velocity is some limit, the limit when the time interval goes to 0. But is it not the change of the displacement in a small time interval? So, we can model it as the \textbf{rate of change of displacement}. We introduce the notation, 
$$\lim_{\Delta t \to 0} \frac{\Delta x}{\Delta t} = \frac{dx}{dt}$$

So, \textbf{derivative is defined as the rate of change of some quantity with respect to another quantity. }


\begin{outline}
    Let us say that we have an independent variable called x and a function of that variable y=f(x)

    Then, how will we define the rate of change of the function with respect to x? We will change the x by a fixed amount and record the f(x) change. Let us say we change the variable x by h amount. Then, the change in the function is f(x+h)-f(x). Thus, the rate of change is defined as:
    $$C = \frac{f(x+h)-f(x)}{h}$$
    As we make it smaller and smaller, we get the derivative of the function f with respect to x. It is written as: 

    $$\frac{df}{dx} = \lim_{h \to 0}\frac{f(x+h)-f(x)}{h}$$
    
\end{outline}


\section{Uses of differentiation}

Differentiation is a mathematical tool that is guaranteed to be helpful in real-world problems because modeling the real world is based on the rate of change, and differentiation is just the mathematical culmination of that. 

\begin{enumerate}
    \item \textbf{Rate of Change Analysis}: Differentiation helps in analyzing how one quantity changes concerning another, essential in fields like physics, economics, and engineering to understand concepts like velocity, acceleration, marginal cost, and revenue.
    
    \item \textbf{Optimization}: It aids in solving optimization problems by finding maximum or minimum values of functions, crucial in engineering for designing efficient systems and structures.
    
    \item \textbf{Modeling with Differential Equations}: Differential equations, involving derivatives, are used extensively in modeling natural phenomena such as population growth, radioactive decay, and motion, providing predictive power in physics, engineering, and other sciences.
    
    \item \textbf{Financial Applications}: In finance, differentiation is used to calculate rates of change of financial quantities like interest rates, stock prices, and option values, facilitating risk assessment, portfolio management, and derivative pricing.
    
    \item \textbf{Image Processing}: Differential calculus plays a vital role in image processing tasks like edge detection, image enhancement, and pattern recognition algorithms, contributing to fields like computer vision and digital image analysis.
    
    \item \textbf{Biological Modeling}: It aids in modeling biological processes such as enzyme kinetics, nerve impulses, and pharmacokinetics, assisting in drug development and understanding physiological systems in biology and medicine.
    
    \item \textbf{Control Systems}: Differentiation is used in control theory to analyze and design feedback control systems, ensuring stability and performance in various engineering applications like robotics, aerospace, and industrial automation.
    
    \item \textbf{Signal Processing}: In fields like telecommunications and audio processing, differentiation helps analyze and manipulate signals, essential for tasks such as filtering, modulation, and noise reduction.
    
    \item \textbf{Machine Learning and Data Analysis}: Differential calculus is utilized in machine learning algorithms for optimization, gradient-based learning, and neural network training, playing a pivotal role in modern data-driven technologies.
    
    \item \textbf{Fluid Dynamics and Heat Transfer}: Differential equations derived from differentiation are used extensively in studying fluid flow and heat transfer phenomena, essential in engineering fields like mechanical, chemical, and aerospace engineering.
\end{enumerate}