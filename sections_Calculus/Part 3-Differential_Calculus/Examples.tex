
This chapter contains exercises that use various techniques discussed earlier; the question solution clarifies which technique has been used. 

\begin{enumerate}
    \item Differentiate the function $f(x)=\sec (\tan (\sqrt{x}))$ with respect to $x$.[Chain Rule]

    \begin{outline}
        Sol. $f(x)=\sec (\tan (\sqrt{x}))$
$$
\begin{aligned}
\therefore \quad f^{\prime}(x) & =\frac{d}{d x}[\sec (\tan \sqrt{x})] \\
& =\sec (\tan \sqrt{x}) \cdot \tan (\tan \sqrt{x}) \cdot \frac{d}{d x}(\tan \sqrt{x}) \\
& =\sec (\tan \sqrt{x}) \cdot \tan (\tan \sqrt{x}) \cdot \sec ^2(\sqrt{x}) \cdot \frac{d}{d x}(\sqrt{x}) \\
& =\sec (\tan \sqrt{x}) \cdot \tan (\tan \sqrt{x}) \cdot \sec ^2(\sqrt{x}) \cdot \frac{1}{2 \sqrt{x}} \\
& =\frac{\sec (\tan \sqrt{x}) \cdot \tan (\tan \sqrt{x}) \sec ^2(\sqrt{x})}{2 \sqrt{x}}
\end{aligned}
$$
    \end{outline}



\item If $y=\sqrt{\log \left\{\sin \left(\frac{x^2}{3}-1\right)\right\}}, \text { then find } \frac{d y}{d x}$.[Chain rule]\\\\

\begin{outline}
    Sol. $y=\sqrt{\log \left\{\sin \left(\frac{x^2}{3}-1\right)\right\}}$
Putting $\frac{x^2}{3}-1=v$, we get $\sin \left(\frac{x^2}{3}-1\right)=\sin v=u$.
Putting $\log \left\{\sin \left(\frac{x^2}{3}-1\right)\right\}=\log u=z$, we get $y=\sqrt{z}, z=\log u$ $u=\sin v$, and $v=\frac{x^2}{3}-1$. Therefore,
$$
\frac{d v}{d z} =\frac{1}{2 \sqrt{z}}, \frac{d z}{d u}=\frac{1}{u}, \frac{d u}{d v}=\cos v, \text { and } \frac{d v}{d x}=\frac{2 x}{3}
$$

Now,

$$
\begin{aligned}
\frac{d y}{d x} & =\frac{d y}{d z} \times \frac{d z}{d u} \times \frac{d u}{d v} \times \frac{d v}{d x} \\
& =\left(\frac{1}{2 \sqrt{z}}\right)\left(\frac{1}{u}\right)(\cos v)\left(\frac{2 x}{3}\right) \\
& =\frac{x}{3} \cdot \frac{\cos v}{u \sqrt{\log u}} \\
& =\frac{x \cot \left(\frac{x^2}{3}-1\right)}{3 \sqrt{\log \left\{\sin \left(\frac{x^2}{3}-1\right)\right\}}}
\end{aligned}
$$
\end{outline}




\item If $y=\sqrt{\frac{1-x}{1+x}}$, prove that $\left(1-x^2\right) \frac{d y}{d x}+y=0$.[Quotient Rule]\\\\


\begin{outline}
    Sol. We have
$$
y=\sqrt{\frac{1-x}{1+x}}
$$

Differentiating w.r.t. $x$, we get
$$
\begin{aligned}
\frac{d y}{d x} & =\frac{1}{2}\left(\frac{1-x}{1+x}\right)^{(1 / 2)-1} \frac{d}{d x}\left(\frac{1-x}{1+x}\right) \\
& =\frac{1}{2} \sqrt{\frac{1+x}{1-x}} \frac{(1+x) \frac{d}{d x}(1-x)-(1-x) \frac{d}{d x}(1+x)}{(1+x)^2} \\
& =\frac{1}{2} \sqrt{\frac{1+x}{1-x}} \frac{(1+x)(-1)-(1-x)(1)}{(1+x)^2}=-\sqrt{\frac{1+x}{1-x}} \frac{1}{(1+x)^2}
\end{aligned}
$$
or $\quad\left(1-x^2\right) \frac{d y}{d x}=-\sqrt{\frac{1+x}{1-x}} \frac{1}{(1+x)^2}\left(1-x^2\right)$\\
or $\quad\left(1-x^2\right) \frac{d y}{d x}=-\sqrt{\frac{1-x}{1+x}}$\\
or $\quad\left(1-x^2\right) \frac{d y}{d x}=-y$\\
or $\quad\left(1-x^2\right) \frac{d y}{d x}+y=0$\\

\end{outline}


\item If $y=x+\frac{1}{x+\frac{1}{x+\frac{1}{x+\cdots}}}$ prove that $\frac{d y}{d x}=\frac{y}{2 y-x}$[Algebraic Manipulation].\\\\


\begin{outline}
    Sol. We have
$$
y=x+\frac{1}{x+\frac{1}{x+\frac{1}{x+\cdots}}}=x+\frac{1}{y}
$$\\
or $y^2=x y+1$\\
or $2 y \frac{d y}{d x}=y+x \frac{d y}{d x}+0 \quad$ [Differentiating both sides w.r.t. $x$ ]\\
or $\quad \frac{d y}{d x}(2 y-x)=y$\\
or $\quad \frac{d y}{d x}=\frac{y}{2 y-x}$\\
\end{outline}

\newpage
\item If $\log \left(x^2+y^2\right)=2 \tan ^{-1}\left(\frac{y}{x}\right)$, show that $\frac{d y}{d x}=\frac{x+y}{x-y}$ [Implicit function handling]  .\\\\

\begin{outline}
    Sol. Differentiating both sides w.r.t. $x$, we get
$$
\begin{aligned}
& \frac{d}{d x}\left\{\log \left(x^2+y^2\right)\right\}=2 \frac{d}{d x}\left\{\tan ^{-1}\left(\frac{y}{x}\right)\right\} \\
& \text { or } \quad \frac{1}{x^2+y^2} \times \frac{d}{d x}\left(x^2+y^2\right)=2 \frac{1}{1+(y / x)^2} \times \frac{d}{d x}\left(\frac{y}{x}\right) \\
& \text { or } \quad \frac{1}{x^2+y^2}\left\{\frac{d}{d x}\left(x^2\right)+\frac{d}{d x}\left(y^2\right)\right\}=2 \times \frac{x^2}{x^2+y^2}\left\{\frac{x \frac{d y}{d x}-y \times 1}{x^2}\right\}\\
& \text{or}  \frac{1}{x^2+y^2}\left\{2 x+2 y \frac{d y}{d x}\right\}=\frac{2}{x^2+y^2}\left\{x \frac{d y}{d x}-y\right\}\\
& \text{or}  \quad 2\left\{x+y \frac{d y}{d x}\right\}=2\left\{x \frac{d y}{d x}-y\right\}\\
& \text{or}  \quad x+y \frac{d y}{d x}=x \frac{d y}{d x}-y\\
& \text{or}  \quad \frac{d y}{d x}(y-x)=-(x+y)\\
& \text{or}  \quad \frac{d y}{d x}=\frac{x+y}{x-y}\\
\end{aligned}
$$

\end{outline}


\item If $x=a \sec ^3 \theta$ and $y=a \tan ^3 \theta$, find $\frac{d y}{d x}$ at $\theta=\frac{\pi}{3}$ [Parametric Functions].\\\\

\begin{outline}
    Sol. We have $x=a \sec ^3 \theta$ and $y=a \tan ^3 \theta$\\
$\therefore \quad \frac{d x}{d \theta}=3 a \sec ^2 \theta \frac{d}{d \theta}(\sec \theta)=3 a \sec ^3 \theta \tan \theta$
and $\frac{d y}{d \theta}=3 a \tan ^2 \theta \frac{d}{d \theta}(\tan \theta)=3 a \tan ^2 \theta \sec ^2 \theta$\\
or $\frac{d y}{d x}=\frac{d y / d \theta}{d x / d \theta}=\frac{3 a \tan ^2 \theta \sec ^2 \theta}{3 a \sec ^3 \theta \tan \theta}=\frac{\tan \theta}{\sec \theta}=\sin \theta$\\
or $\quad\left(\frac{d y}{d x}\right)_{\theta=\pi / 3}=\sin \frac{\pi}{3}=\frac{\sqrt{3}}{2}$

\end{outline}


\item If $x^m y^n=(x+y)^{m+n}$, prove that $\frac{d y}{d x}=\frac{y}{x}$.


\begin{outline}
    Sol. We have $x^m y^n=(x+y)^{m+n}$ [Logarithmic differentiation].\\\\
Taking log on both sides, we get
$$
m \log x+n \log y=(m+n) \log (x+y)
$$

Differentiating both sides w.r.t. $x$, we get\\
$m \frac{1}{x}+n \frac{1}{d x}=\frac{m+n}{x+y} \frac{d}{d x}(x+y)$ or $\frac{m}{x}+\frac{n}{y} \frac{d y}{d x}=\frac{m+n}{x+y}\left(1+\frac{d y}{d x}\right)$\\
or $\quad\left\{\frac{n}{y}-\frac{m+n}{x+y}\right\} \frac{d y}{d x}=\frac{m+n}{x+y}-\frac{m}{x}$\\
or $\quad\left\{\frac{n x+n y-m y-n y}{y(x+y)}\right\} \frac{d y}{d x}=\left\{\frac{m x+n x-m x-m y}{(x+y) x}\right\}$\\
or $\frac{n x-m y}{y(x+y)} \frac{d y}{d x}=\frac{n x-m y}{(x+y) x}$\\
or $\quad \frac{d y}{d x}=\frac{y}{x}$
\end{outline}



\item Differentiate $(\log x)^{\cos x}$ with respect to $x$.


\begin{outline}
    Sol. Let $y=(\log x)^{\cos x}$[Logarithmic differentiation].\\\\
Taking logarithms on both sides, we obtain
$$
\log y=\cos x \cdot \log (\log x)
$$

Differentiating both sides with respect to $x$, we obtain
$$
\begin{aligned}
\frac{1}{y} \cdot \frac{d y}{d x} & =\log (\log x) \frac{d}{d x}(\cos x)+\cos x \cdot \frac{d}{d x}[\log (\log x)] \\
\text { or } \frac{1}{y} \cdot \frac{d y}{d x} & =-\sin x \log (\log x)+\cos x \cdot \frac{1}{\log x} \cdot \frac{d}{d x}(\log x) \\
\text { or } \frac{d y}{d x} & =y\left[-\sin x \log (\log x)+\frac{\cos x}{\log x} \cdot \frac{1}{x}\right] \\
& =(\log x)^{\cos x}\left[\frac{\cos x}{x \log x}-\sin x \log (\log x)\right]
\end{aligned}
$$
\end{outline}


\item Differentiate $\tan ^{-1}\left(\frac{\sqrt{1+x^2}-1}{x}\right)$ w.r.t. $\tan ^{-1} x$, where $x \neq 0$.\\\\


\begin{outline}
    Sol. Let $u=\tan ^{-1}\left(\frac{\sqrt{1+x^2}-1}{x}\right)$ and $v=\tan ^{-1} x$.
Putting $x=\tan \theta$, we get
$$
\begin{aligned}
u & =\tan ^{-1}\left(\frac{\sqrt{1+x^2}-1}{x}\right) \\
& =\tan ^{-1}\left(\frac{\sec \theta-1}{\tan \theta}\right) \\
& =\tan ^{-1}\left(\frac{1-\cos \theta}{\sin \theta}\right) \\
& =\tan ^{-1}\left(\tan \frac{\theta}{2}\right) \\
& =\frac{1}{2} \theta \\
& =\frac{1}{2} \tan ^{-1} x
\end{aligned}
$$

Thus, we have $u=\frac{1}{2} \tan ^{-1} x$ and $v=\tan ^{-1} x$. Therefore,
$$
\begin{aligned}
\frac{d u}{d x} & =\frac{1}{2} \times \frac{1}{1+x^2} \text { and } \frac{d v}{d x}=\frac{1}{1+x^2} \\
\therefore \quad \frac{d u}{d v} & =\frac{d u / d x}{d v / d x}=\frac{1}{2\left(1+x^2\right)}\left(1+x^2\right)=\frac{1}{2}
\end{aligned}
$$
\end{outline}


\item If $(x-a)^2+(y-b)^2=c^2$, for some $c>0$, prove that $\frac{\left[1+\left(\frac{d y}{d x}\right)^2\right]^{\frac{3}{2}}}{\frac{d^2 y}{d x^2}}$ is a constant or independent of $a$ and $b$ \\\\

\begin{outline}
    Sol. We have $(x-a)^2+(y-b)^2=c^2, c>0$
Differentiating w.r.t. $x$, we get
$$2(x-a)+2(y-b) y^{\prime}=0$$ $$ \text{or}\quad(x-a)+(y-b) y^{\prime}=0 \quad \quad (1)$$
$$\text{or} \quad y^{\prime}=-\frac{x-a}{y-b}$$
Differentiating (1), w.r.t. $x$ again, we get
$$
\begin{aligned}
& 1+\left(y^{\prime}\right)^2+(y-b) y^{\prime \prime}=0 \\
\therefore \quad & \frac{\left[1+\left(\frac{d y}{d x}\right)^2\right]^{\frac{3}{2}}}{\frac{d^2 y}{d x^2}}=\frac{\left[1+\left(y^{\prime}\right)^2\right]^{\frac{3}{2}}}{\frac{-\left[1+\left(y^{\prime}\right)^2\right]}{y-b}} \\
& =-(y-b)\left[1+\left(y^{\prime}\right)^2\right]^{\frac{1}{2}} \\
& =-(y-b)\left[1+\left(\frac{x-a}{y-b}\right)^2\right]^{\frac{1}{2}} \\
& =-\left[(y-b)^2+(x-a)^2\right]^{\frac{1}{2}} \\
& =-c
\end{aligned}
$$

\end{outline}





\item If
If $x=\operatorname{cosec} \theta-\sin \theta$ and $y=\operatorname{cosec}^n \theta-\sin ^n \theta$, then show that
$$
\left(x^2+4\right)\left(\frac{d y}{d x}\right)^2=n^2\left(y^2+4\right) \text {. }
$$\\\\


\begin{outline}
    Sol. As $x=\operatorname{cosec} \theta-\sin \theta$, we have
$$
\begin{aligned}
& x^2+4=(\operatorname{cosec} \theta-\sin \theta)^2+4=(\operatorname{cosec} \theta+\sin \theta)^2 \\
& y^2+4=\left(\operatorname{cosec}^n \theta-\sin ^n \theta\right)^2+4=\left(\operatorname{cosec}^n \theta+\sin ^n \theta\right)^2
\end{aligned}
$$
and
Now.
$$
\begin{aligned}
\frac{d y}{d x} & =\frac{\left(\frac{d y}{d \theta}\right)}{\left(\frac{d x}{d \theta}\right)}=\frac{n\left(\operatorname{cosec}^{n-1} \theta\right)(-\operatorname{cosec} \theta \cot \theta)-n \sin ^{n-1} \theta \cos \theta}{-\operatorname{cosec} \theta \cot \theta-\cos \theta} \\
& =\frac{n\left(\operatorname{cosec}^n \theta \cot \theta+\sin ^{n-1} \theta \cos \theta\right)}{(\operatorname{cosec} \theta \cot \theta+\cos \theta)} \\
& =\frac{n \cot \theta\left(\operatorname{cosec} ^n \theta+\sin ^n \theta\right)}{\cot \theta(\operatorname{cosec} \theta+\sin \theta)} \\
& =\frac{n\left(\operatorname{cosec}{ }^n \theta+\sin ^n \theta\right)}{(\operatorname{cosec} \theta+\sin \theta)}=\frac{n \sqrt{y^2+4}}{\sqrt{x^2+4}} \quad \text { [From (1) and (2)] }
\end{aligned}
$$
[From (1) and (2)]

Squaring both sides, we get $\left(\frac{d y}{d x}\right)^2=\frac{n^2\left(y^2+4\right)}{\left(x^2+4\right)}$
of $\left(x^2+4\right)\left(\frac{d y}{d x}\right)^2=n^2\left(y^2+4\right)$
\end{outline}


 \item If $u=f\left(x^2\right), \quad v=g\left(x^3\right), f^{\prime}(x)=\sin x$ and $g^{\prime}(x)=\cos x$ then find $\frac{d u}{d v}$.\\\\


 \begin{outline}
     Differentiating $u=f\left(x^2\right)$ and $v=g\left(x^3\right)$ w.r.t. $x$ we get
$$
\frac{d u}{d x}=f^{\prime}\left(x^2\right)\cdot 2 x=\sin \left(x^2\right) \cdot 2 x
$$
$$
\begin{gathered}
\left.\because f^{\prime}(x)=\sin x \Rightarrow f^{\prime}\left(x^2\right)=\sin \left(x^2\right)\right\} \\
\frac{d v}{d x}=g^{\prime}\left(x^3\right) \cdot 3 x^2=\cos \left(x^3\right) \cdot 3 x^2
\end{gathered}
$$
$$
\left\{\because g^{\prime}(x)=\cos x \Rightarrow g^{\prime}\left(x^3\right)=\cos \left(x^3\right)\right\}
$$
$$
\therefore \frac{d u}{d v}=\frac{\frac{d u}{d x}}{\frac{d v}{d x}}=\frac{\sin \left(x^2\right) \cdot 2 x}{\cos \left(x^3\right) \cdot 3 x^2}=\frac{2}{3 x} \cdot \frac{\sin x^2}{\cos x^3} \text {. }
$$


 \end{outline}

\item Let $f(x)$ and $g(x)$ be two functions having finite nonzero third order derivatives $f^{\prime \prime \prime}(x)$ and $g^{\prime \prime \prime}(x)$ for all $x \in R$. If $f(x) g(x)$ $=1$ for all $x \in R$, then prove that $\frac{f^{\prime \prime \prime}}{f^{\prime}}-\frac{g^{\prime \prime \prime}}{g^{\prime}}=3\left(\frac{f^{\prime \prime}}{f}-\frac{g^{\prime \prime}}{g}\right)$.\\\\



\begin{outline}
    Sol. We have $f(x) g(x)=1$. Differentiating with respect to $x$, we get
$$
f^{\prime} g+f g^{\prime}=0
 \quad \quad \quad 1$$

Differentiating (1) w.r.t. $x$, we get
$$
f^{\prime \prime} g+2 f^{\prime} g^{\prime}+f g^{\prime \prime}=0
\quad \quad \quad (2)$$

Differentiating (2) w.r.t. $x$, we get
or
$$
\begin{aligned}
& f^{\prime \prime \prime} g+g^{\prime \prime \prime} f+3 f^{\prime \prime} g^{\prime}+3 g^{\prime \prime} f^{\prime}=0 \\
& \implies \frac{f^{\prime \prime \prime}}{f^{\prime}}\left(f^\prime g\right)+\frac{g^{\prime \prime \prime}}{g^{\prime}}\left(f g^\prime\right)+\frac{3 f^{\prime \prime}}{f}\left(fg^{\prime} \right)+\frac{3 g^{\prime \prime}}{g}\left(f^\prime g\right)=0
\end{aligned}
 \quad \quad \text{[Using (1)]}$$

or $\quad\left(\frac{f^{\prime \prime \prime}}{f^{\prime}}+\frac{3 g^{\prime \prime}}{g}\right)\left(f g^{\prime}\right)=-\left(\frac{g^{\prime \prime \prime}}{g^{\prime}}+\frac{3 f^{\prime \prime}}{f}\right)\left(f g^{\prime}\right)$

$$  \text{or} \quad
-\left(\frac{f^{\prime \prime \prime}}{f^{\prime}}+\frac{3 g^{\prime \prime}}{g}\right)\left(f g^{\prime}\right)=\left(\frac{g^{\prime \prime \prime}}{g^{\prime}}+\frac{3 f^{\prime \prime}}{g}\right) f g^{\prime}
$$
[Using (1)]
or $\frac{f^{\prime \prime \prime}}{f^{\prime}}+\frac{3 g^{\prime \prime}}{g}=\frac{g^{\prime \prime \prime}}{g^{\prime}}+\frac{3 f^{\prime \prime}}{f}$
or $\quad \frac{f^{\prime \prime \prime}}{f^{\prime}}-\frac{g^{\prime \prime \prime}}{g^{\prime}}=3\left(\frac{f^{\prime \prime}}{f}-\frac{g^{\prime \prime}}{g}\right)$
\end{outline}

\newpage
\item
Find $\frac{d y}{d x}$ at $x=-1$ when
$$
(\sin y)^{\sin \frac{\pi x}{2}}+\frac{\sqrt{3}}{2} \sec ^{-1}(2 x)+2^x \tan \left(\log _e(x+2)\right\}=0 .
$$

\begin{outline}
    We know,
$\frac{d\left(\bar{u}^v\right)}{d x}=$ diff. coeff. of $u^v$ w.rt. $x$ when $v$ is a constant
+ diff. coeff. of $u^0$ w.r.t. $x$ when $u$ is a constant.
$\therefore$ differentiating the given relation w.r.t. $x$,
$$
\begin{aligned}
& \sin \frac{\pi x}{2} \cdot(\sin y)^{\sin \frac{x x}{2}-1} \cdot \cos y \frac{d y}{d x} \\
& +(\sin y)^{\sin \frac{\pi x}{2}} \cdot \log \sin y, \frac{\pi}{2} \cos \frac{\pi x}{2} \\
& +\frac{\sqrt{3}}{2}\cdot \frac{1}{|2 x| \sqrt{4 x^2-1}} \cdot 2 \\
& +2^x \log 2 \cdot \tan \left\{\log _e(x+2)\right\} \\
& +2^x \cdot \sec ^2\left[\log _e(x+2)\right] \cdot \frac{1}{x+2}=0 . \\
&
\end{aligned}
$$

Putting $x=-1$, we get (assuming $y=y_0$ wheri $x=-1$ )
$$
\begin{aligned}
-\left(\sin y_0\right)^{-2} \cdot \cos y_0 & \cdot\left[\frac{d y}{d x}\right]_{x=-1} \\
+\left(\sin y_0\right)^{-1} & \cdot \log \sin y_0 \cdot \frac{\pi}{2} \cdot 0 \\
& +\frac{\sqrt{3}}{2} \cdot \frac{1}{1-21 \sqrt{3}} \cdot 2+0+\frac{1}{2}+1 \cdot \frac{1}{1}=0
\end{aligned}
$$
or $-\cos y_0\left(\sin y_0\right)^{-2} \cdot\left[\frac{d y}{d x}\right]_{x=-1}+\frac{1}{2}+\frac{1}{2}=0$
$\therefore\left[\frac{d y}{d x}\right]_{x=-1}=\frac{\sin ^2 y_0}{\cos y_0}$
Now, putting $x=-1$ in the original relation,
$$
\begin{aligned}
& \left(\sin y_0\right)^{-1}+\frac{\sqrt{3}}{2} \sec ^{-1}(-2)+\frac{1}{2} \tan \log _e 1=0 \\
& \text { or } \quad\left(\sin y_0\right)^{-1}+\frac{\sqrt{3}}{2}\left(\pi-\frac{\pi}{3}\right)=0 \\
& \therefore \quad \frac{1}{\sin y_0}+\frac{\pi}{\sqrt{3}}=0 \\
& \therefore \quad \sin y_0=-\frac{\sqrt{3}}{\pi} . \\
& \text { Hence, }\left[\frac{d y}{d x}\right]_{x=-1}=\frac{\left(-\frac{\sqrt{3}}{\pi}\right)^2}{\sqrt{1-\left(\frac{\sqrt{3}}{\pi}\right)^2}} \\
& =\frac{3}{\pi \sqrt{\pi^2-3}} .
\end{aligned}
$$

\end{outline}

\newpage

\item If $\sqrt{1-x^{2 n}}+\sqrt{1-y^{2 n}}=a^n\left(x^n-y^n\right)$, prove that
$$
y^{n-1} \cdot \sqrt{1-x^{2 n}} d y=x^{n-1} \sqrt{1-y^{2 n}} d x .
$$\\\\



\begin{outline}
    Differentiating both sides w.r.t. $x$,
$$
\begin{aligned}
& \frac{1}{2} \cdot \frac{1}{\sqrt{1-x^{2 n}}} \cdot\left(-2 n x^{2 n-1}\right) \\
& \quad+\frac{1}{2} \cdot \frac{1}{\sqrt{1-y^{2 n}}}\left(-2 n y^{2 n-1}\right) \frac{d y}{d x} \\
& =a^n\left\{n x^{n-1}-n y^{n-1} \frac{d y}{d x}\right\}
\end{aligned}
$$
 $$ \text{or} \quad \frac{x^{2 n-1}}{\sqrt{1-x^{2 n}}}+\frac{y^{2 n-1}}{\sqrt{1-y^{2 n}}} \frac{d y}{d x}=a^n\left(y^{n-1} \frac{d y}{d x}-x^{n-1}\right)$$
$$\text{or} \quad\frac{x^{2 n-1}}{\sqrt{1-x^{2 n}}}+a^n x^{n-1}=\left(a^n y^{n-1}-\frac{y^{2 n-1}}{\sqrt{1-y^{2 n}}}\right) \frac{d y}{d x}$$
$$ \text{or}\quad x^{n-1}, \frac{x^n+a^n \cdot \sqrt{1-x^{2 n}}}{\sqrt{1-x^{2 n}}} =y^{n-1} \cdot \frac{a^n \cdot \sqrt{1-y^{2 n}}-y^n}{\sqrt{1-y^{2 n}}} \cdot \frac{d y}{d x}
$$
$$\text{or} \quad \frac{d y}{d x}=\frac{x^{n-1}}{y^{n-1}} \cdot \sqrt{\frac{1-y^{2 n}}{1-x^{2 n}}} \cdot \frac{x^n+a^n \cdot \sqrt{1-x^{2 n}}}{a^n \cdot \sqrt{1-y^{2 n}}-y^n}$$
Now, we have
$$
\sqrt{1-x^{2 n}}+\sqrt{1-y^{2 n}}=a^n\left(x^n-y^n\right)
$$
 $$\text{or} \quad \left(1-x^{2 n}\right)-\left(1-y^{2 n}\right)
=a^n\left(x^n-y^n\right)\left(\sqrt{1-x^{2 n}}-\sqrt{1-y^{2 n}}\right)
$$
 $$ \text{or} \quad y^{2 n}-x^{2 n}=a^n\left(x^n-y^n\right)\left\{\sqrt{1-x^{2 n}}-\sqrt{1-y^{2 n}}\right)$$
$$\therefore \sqrt{1-x^{2 n}}-\sqrt{1-y^{2 n}}=\frac{-1}{a^n}\left(y^n+x^n\right)$$
 $$ \text{or} a^n \sqrt{1-x^{2 n}}+x^n=a^n \sqrt{1-y^{2 n}}-y^n$$
Using (2) in (1),
$$
\frac{d y}{d x}=\frac{x^{n-1}}{y^{n-1}} \cdot \sqrt{\frac{1-y^{2 n}}{1-x^{2 n}}} .
$$

Writing in the differential form,
$$
y^{n-1} \cdot \sqrt{1-x^{2 n}} d y=x^{n-1} \sqrt{1-y^{2 n}} d x .
$$

\end{outline}

\newpage
\item 
If $y=\frac{a x^2}{(x-a)(x-b)(x-c)}+\frac{b x}{(x-b)(x-c)}+\frac{c}{x-c}+1$, prove that $\frac{y^{\prime}}{y}=\frac{1}{x}\left(\frac{a}{a-x}+\frac{b}{b-x}+\frac{c}{c-x}\right)$.\\\\



\begin{outline}
    Sol.
$$
\begin{aligned}
y & =\frac{a x^2}{(x-a)(x-b)(x-c)}+\frac{b x}{(x-b)(x-c)}+\frac{c}{x-c}+1 \\
& =\frac{a x^2}{(x-a)(x-b)(x-c)}+\frac{b x}{(x-b)(x-c)}+\left(\frac{c+x-c}{x-c}\right)
\end{aligned}
$$
$$
\begin{aligned}
& =\frac{a x^2}{(x-a)(x-b)(x-c)}+\frac{b x}{(x-b)(x-c)}+\frac{x}{x-c} \\
& =\frac{a x^2}{(x-a)(x-b)(x-c)}+\frac{b x+x(x-b)}{(x-b)(x-c)} \\
& =\frac{a x^2}{(x-a)(x-b)(x-c)}+\frac{x^2}{(x-b)(x-c)} \\
& =\frac{a x^2+x^2(x-a)}{(x-a)(x-b)(x-c)} \\
& =\frac{x^3}{(x-a)(x-b)(x-c)} \\
& \therefore \quad \log y=\log \left\{\frac{x^3}{(x-a)(x-b)(x-c)}\right\} \\
& \text { or } \log y=3 \log x-\{\log (x-a)+\log (x-b)+\log (x-c)\} \\
& \frac{1}{y} \frac{d y}{d x}=\frac{3}{x}-\left\{\frac{1}{x-a}+\frac{1}{x-b}+\frac{1}{x-c}\right\} \\
& \text { or } \frac{d y}{d x}=y\left\{\left(\frac{1}{x}-\frac{1}{x-a}\right)+\left(\frac{1}{x}-\frac{1}{x-b}\right)+\left(\frac{1}{x}-\frac{1}{x-c}\right)\right\} \\
& =y\left\{-\frac{a}{x(x-a)}-\frac{b}{x(x-b)}-\frac{c}{x(x-c)}\right\} \\
& =\frac{y}{x}\left\{\frac{a}{a-x}+\frac{b}{b-x}+\frac{c}{x-c}\right\} \text {. } \\
&
\end{aligned}
$$

On differentiating w.r.t. $x$, we get
$$
\frac{1}{y} \frac{d y}{d x}=\frac{3}{x}-\left\{\frac{1}{x-a}+\frac{1}{x-b}+\frac{1}{x-c}\right\}
$$
\end{outline}




\end{enumerate}




\section{Exercises}

Now, there are miscellaneous exercises.

Find the derivative of $y$ w.r.t. $x$.

\begin{enumerate}
    \item $y=\sin ^{-1} \frac{2 x}{1+x^2},-1 \leq \mathrm{x} \leq 1$
    \item $y=\tan ^{-1} \frac{3 x-x^3}{1-3 x^2},-\frac{1}{\sqrt{3}}<x<\frac{1}{\sqrt{3}}$
    \item $y=\sec ^{-1} \frac{1}{2 x^2-1}, 0<x<\frac{1}{\sqrt{2}}$
    \item $y=\tan ^{-1} \frac{4 x}{1+5 x^2}+\tan ^{-1} \frac{2+3 x}{3-2 x}$
    \item $y=\tan ^{-1}\left(\frac{\sqrt{1+x^2}-1}{x}\right), x \neq 0$
    \item $y=\tan ^{-1}\left(\frac{x}{1+\sqrt{1-x^2}}\right)$
    \item $y=\sin ^{-1} \sqrt{(1-x)}+\cos ^{-1} \sqrt{x}$
    \item $y=\sqrt{\sin \sqrt{x}}$
    \item $y=e^{\sin x^2}$
    \item $y=\log \sqrt{\sin \sqrt{e^x}}$
    \item $y=a^{\left(\sin ^{-1} x\right)^2}$
    \item $y=\log \left\{e^x\left(\frac{x-2}{x+2}\right)^{3 / 4}\right\}$
    \item $y=\sin ^{-1}[\sqrt{x-a x}-\sqrt{a-a x}]$
    \item $y=x^3 e^x \sin x$
    \item $y=\log _e \sqrt{\frac{1+\sin x}{1-\sin x}}$, where $x=\pi / 3$
    \item $y=\frac{x+\sin x}{x+\cos x}$
    \item If $y=(1+x)\left(1+x^2\right)\left(1+x^4\right) \cdots\left(1+x^{2^n}\right)$, then find $\frac{d y}{d x}$ at $x=0$.
    \item If $x \sqrt{1+y}+y \sqrt{1+x}=0$, prove that $\frac{d y}{d x}=-\frac{1}{(x+1)^2}$.
    \item If $g$ is the inverse function of $f$ and $f^{\prime}(x)=\sin x$ then prove that $g^{\prime}(x)=\operatorname{cosec}(g(x))$
    \item If $y=b \tan ^{-1}\left(\frac{x}{a}+\tan ^{-1} \frac{y}{x}\right)$, find $\frac{d y}{d x}$.
    \item If $\log _e\left(\log _e x-\log _e y\right)=e^{x^2 y}\left(1-\log _e x\right)$, then find the value of $y^{\prime}(e)$
    \item If $y=\sqrt{x+\sqrt{y+\sqrt{x+\sqrt{y+\ldots \infty}}}}$, then prove that $\frac{d y}{d x}=\frac{y^2-x}{2 y^3-2 x y-1}$
    \item If $x=\frac{2 t}{1+t^2}, y=\frac{1-t^2}{1+t^2}$, then find $\frac{d y}{d x}$ at $t=2$.
    \item If $x=\sqrt{a^{\sin ^{-1} t}}, y=\sqrt{a^{\cos ^{-1} t}}, a>0$ and $-1<t<1$, show that $\frac{d y}{d x}=-\frac{y}{x}$.
    \item Find $\frac{d y}{d x}$, if $x=\cos \theta-\cos 2 \theta$ and $y=\sin \theta-\sin 2 \theta$.
    \item Find $\frac{d y}{d x}$ if $x=3 \cos \theta-2 \cos ^3 \theta, y=3 \sin \theta-2 \sin ^3 \theta$.
    \item Find $\frac{d y}{d x}$ at $t=\pi / 4$ for $x=a\left[\cos t+\frac{1}{2} \log \tan ^2 \frac{t}{2}\right]$ and $y=a \sin t$.
    \item Differentiate $\sqrt{\frac{(x-1)(x-2)}{(x-3)(x-4)(x-5)}}$ with respect to $x$.
    \item If $x^y=e^{x-y}$, prove that $\frac{d y}{d x}=\frac{\log x}{(1+\log x)^2}$.
    \item If $x y=e^{(x-y)}$, then find $\frac{d y}{d x}$.
    \item If $y^x=x^y$, then find $\frac{d y}{d x}$.
    \item If $x=e^{y+e^{y+\ldots \infty}}$, where $x>0$, then find $\frac{d y}{d x}$.
    \item Find $\frac{d y}{d x}$ for $y=x^x$.
    \item Differentiate $(x \cos x)^x$ with respect to $x$.
    \item If $y=(\tan x)^{(\tan x)^{\tan x}}$, then find $\frac{d y}{d x}$.
    \item If $f(x)=(1+x)^n$, then find the value of $f(0)+f^{\prime}(0)+\frac{f^{\prime \prime}(0)}{2 !}+\frac{f^{\prime \prime \prime}(0)}{3 !}+\cdots+\frac{f^{\prime \prime n}(0)}{n !}$.
    \item If $e^x(x+1)=1$, show that $\frac{d^2 y}{d x^2}=\left(\frac{d y}{d x}\right)^2$.
    \item Prove that $\frac{d^{n}}{d x^{n}}\left(e^{2 x}+e^{-2 t}\right)=2^{n}\left[e^{2 x}+(-1)^{n} e^{-2 x}\right]$.
    \item If $y=\sin (\sin x)$ and $\frac{d^2 y}{d x^2}+\frac{d y}{d x} \tan x+f(x)=0$, then find $f(x)$.
    \item If $y=\log (1+\sin x)$, prove that $y_4+y_3 y_1+y_2^2=0$. of $\frac{d^n}{d x^n}[f(x)]_{x=0}$
    \item If $x=a \cos \theta, y=b \sin \theta$, then prove that $\frac{d^3 y}{d x^3}=-\frac{3 b}{a^3} \operatorname{cosec}^4 \theta \cot \theta$.
    \item If $x=a \cos ^3 \theta, y=b \sin ^3 \theta$, find $\frac{d^3 y}{d^3}$ at $\theta=0$.
\end{enumerate}

\textbf{Some more Advanced Problems}

\begin{enumerate}

\item If $y=\frac{x^2}{2}+\frac{x \sqrt{x^2+1}}{2}+\log \sqrt{x+\sqrt{x^2+1}}$, prove that $2 y=x \frac{d y}{d x}+\log \left(\frac{d y}{d x}\right)$.
\item If $y=1+\frac{a}{x-a}+\frac{b x}{(x-a)(x-b)}+\frac{c x^2}{(x-a)(x-b)(x-c)}$, show that $\frac{d y}{d x}=\frac{y}{x}\left|\frac{a}{a-x}+\frac{b}{b-x}+\frac{c}{c-x}\right|$.
\item If $y=\sqrt{a^2-x^2}+\frac{a}{2} \log \frac{a-\sqrt{a^2-x^2}}{a+\sqrt{a^2-x^2}}$, show that $\frac{d x}{d y}=\frac{x}{\sqrt{a^2-x^2}}$.

    \item $x^{\sin y}+y^{\cos x}=1$
    \item $x^y+y^x=(x+y)^{x+y}$\frac{x}{a+} \frac{x}{b+} \ldots$
    \item $y=(\sqrt{} x)^{x^{x^... \infty}}$
    \item If $\sqrt{1-x^4}+\sqrt{1-y^4}=k\left(x^2-y^2\right)$, prove that
    \[
    \frac{d y}{d x}=\frac{x \sqrt{1-y^4}}{y \sqrt{1-x^4}} \text {. }
    \]
    \item Find the derivative of
    \[
    f(x)=\log _x \sin x^2+\left(\sin x^2\right)^{\log x} x^x
    \]
    w.r.t. $\phi(x)=\log _e x$.
    \item Find the differential coefficient of $\sec ^{-1} \frac{1}{2 x^2-1}$ w.r.t. $\sqrt{1-x^2}$ at $x=\frac{1}{2}$.
    \item Find the differential coefficient of $\log _{(1-\sqrt{x})} \sin ^{-1}(1-\sqrt{x})$ with respect to $2^{2(1-\sqrt{x})}$ and also find its value at $x=0.25$.
    \item If $y=\sqrt{\frac{t-\alpha}{\beta-t}}$ and $x=\sqrt{(t-\alpha)(\beta-t)}$ then find $\frac{d y}{d x}$.
    \item If $x=f(t)$ and $y=\phi(t)$, prove that
    \[
    \frac{d^2 y}{d x^2}-\frac{f_1 \phi_1-f_2 \phi_1}{f_1}
    \]
    where suffixes denote differentiation w.r.t. $t$.
    \item If $x=\sec \theta-\cos \theta$ and $y=\sec ^n \theta-\cos ^n \theta$, show that
    \[
    \left(4+x^2\right)\left(\frac{d y}{d x}\right)^2=n^2\left(4+y^2\right) \text {. }
    \]
    \item If $x=\cos \theta, y=\sin ^3 \theta$ then show that
    \[
    \begin{aligned}
    & x=\cos \theta, y=\sin ^2 \theta \text { then show that } \\
    & y \frac{d^2 y}{d x^2}+\left(\frac{d y}{d x}\right)^2=3 \sin ^2 \theta \cdot\left(5 \cos ^2 \theta-1\right) . \quad n \text { if }
    \end{aligned}
    \]
    \item If $y=2 \cos t-\cos 2 t$ and $x=2 \sin t-\sin 2 t$ then prove that $\left(\frac{d^2 y}{d x^2}\right)_{t=n / 2}=\frac{-3}{2}$.
    \item Find $\frac{d y}{d x}$ when $\sqrt{1-y^2}+\sqrt{1-t^2}=\alpha(y-t)$ and $x=\sin ^{-1}\left(t \cdot \sqrt{1-t}+\sqrt{t} \cdot \sqrt{1-t^2}\right)$. Express your result as a function of $y$ and $t$ independent of $\alpha$.
    \item If $x^2-y^2=t-\frac{1}{t}$ and $x^4+y^4=t^2+\frac{1}{t^2}$ then prove that $x^3 y \frac{d y}{d x}+1=0$.
    \item Prove that $\frac{d}{d x} \cdot\left|\begin{array}{lll}u_1 & v_1 & w_1 \\ u_2 & v_2 & w_2 \\ u_3 & v_3 & w_3\end{array}\right|=\left|\begin{array}{lll}u_1 & v_1 & w_1 \\ u_2 & v_2 & w_2 \\ u_4 & v_4 & w_4\end{array}\right|$ where $u, v, w$ are functions of $x$ and $\frac{d u}{d x}=u_1, \frac{d^2 u}{d x^2}=u_2$ etc.
    \item If $f_r(x), g_r(x), h_r(x) ; r=1,2,3$ are polynomials in $x$ such that $f_r(a)=g_r(a)=h_r(a) ; r=1,2,3$ and
    \[
    F(x)=\left|\begin{array}{lll}
    f_1(x) & f_2(x) & f_3(x) \\
    g_1(x) & g_2(x) & g_3(x) \\
    h_1(x) & h_2(x) & h_3(x)
    \end{array}\right| \text { then find } F^{\prime}(a) \text {. }
    \]
    \item If $y=\cos a x$, show that $\left|\begin{array}{lll}y & y_1 & y_2 \\ y_3 & y_4 & y_5 \\ y_6 & y_7 & y_8\end{array}\right|=0$ where $y_r=\frac{d^r}{d x^r}(\cos a x)$.
    \item If $\$(x)=\lambda(x) \cdot f(x)$ and $\psi(x)=\mu(x) \cdot f(x)$ then prove that
    \[
    \left|\begin{array}{ccc}
    f(x) & \phi(x) & \Psi(x) \\
    f^{\prime}(x) & \phi^{\prime}(x) & \psi^{\prime \prime}(x) \\
    f^{\prime \prime}(x) & \phi^{\prime \prime}(x) & \psi^{\prime \prime}(x)
    \end{array}\right|=\left\langle\left. f(x)\right|^3 \cdot\left|\begin{array}{cc}
    \lambda^{\prime}(x) & \mu^{\prime}(x) \\
    \lambda^{\prime \prime}(x) & \mu^{\prime \prime}(x)
    \end{array}\right| .\right.
    \]
    \item Let $\Delta(x)=\left|\begin{array}{ccc}x^2-1 & x+1 & x-2 \\ 2 x^2-1 & 3 x & 3 x-3 \\ x^2+4 & 2 x-1 & 2 x-1\end{array}\right|$. Prove, by using calculus, that $\Delta(x)$ is a first degree polynomial.
    \item If $y=\sin \left(2 \sin ^{-1} x\right)$, show that
    \[
    \left(1-x^2\right) \frac{d^2 y}{d x^2}=x \frac{d y}{d x}-4 y .
    \]
    \item If $y=a \cos (\log x)+b \sin (\log x)$ where $a, b$ are arbitrary constants then obtain the relation between $x, y$, $\frac{d y}{d x}$ and $\frac{d^2 y}{d x^2}$ which does not contain $a$ and $b$.

    \item If $y=x^{n-1} \log x$, prove that
$$
x^2 y_2+(3-2 n) x y_1+(n-1)^2 \cdot y=0
$$
where $y_1=\frac{d y}{d x}$ and $y_2=\frac{d^2 y}{d x^2}$.
\item Let $y=\tan ^{-1} \sqrt{x^2-1}$. Prove that
$$
\left(2 x^2-1\right) \frac{d y}{d x}+x\left(x^2-1\right) \frac{d^2 y}{d x^2}=0 .
$$
\item If $y=\frac{a x+b}{x^2+c}$, prove that $\left(2 x y_1+y\right) y_3=3\left(x y_2+y_1\right) y_2$.
\item Let $y=f(x) \cdot \phi(x)$ and $z=f^{\prime}(x) \cdot \phi^{\prime}(x)$.
Prove that
$$
\frac{1}{y} \cdot \frac{d^2 y}{d x^2}=\frac{1}{f} \cdot \frac{d^2 f}{d x^2}+\frac{1}{\phi}, \frac{d^2 \phi}{d x^2}+\frac{2 z}{f \phi} \text {. }
$$
\end{enumerate}
