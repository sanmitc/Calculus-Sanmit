\section{Examples}

Now that we have what theoretically means for a function to be continuous let us look at some examples. These examples give you a better insight into how to test continuity and a more profound intuition of this concept. You will be able to test continuity with limits. But it is also possible to decompose complex functions in more straightforward functions and extrapolate from their behaviour.

\begin{enumerate}
    \item What type of discontinuity does $f(x)=\frac{1}{\log _e\left|x^2-4\right|}$ have?\\\\

\begin{outline}
    Sol. We have
$$
f(x)=\frac{1}{\log _e\left|x^2-4\right|}
$$

Clearly $f(x)$ is defined if $\log _e\left|x^2-4\right|>0$ and $\left|x^2-4\right| \neq 1$
$$
\begin{aligned}
& \therefore \quad x^2-4 \neq 0 \text { and } x^2-4 \neq 1,-1 \\
& \Rightarrow \quad x^2 \neq 4 \text { and } x^2 \neq 3,5 \\
& \Rightarrow \quad x \neq \pm 2 \text { and } x \neq \pm \sqrt{5}, \pm \sqrt{3} \\
& \text { Now } \quad \lim _{x \rightarrow \pm 2} \frac{1}{\log _e\left|x^2-4\right|}=\lim _{y \rightarrow 0} \frac{1}{\log _e y}=0
\end{aligned}
$$

Thus, $f(x)$ has missing point type of discontinuity at $x= \pm 2$ which is removable type.
$$
\lim _{x \rightarrow \sqrt{3}^{+}} \frac{1}{\log _e\left|x^2-4\right|}=\lim _{y \rightarrow 1^{-}} \frac{1}{\log _e y}=-\infty
$$

Thus, $f(x)$ has infinity type of discontinuity at $x=\sqrt{3}$, which is nonremovable type.
Also $\lim _{x \rightarrow \sqrt{5}^{+}} \frac{1}{\log _e\left|x^2-4\right|}=\lim _{y \rightarrow 1^{+}} \frac{1}{\log _e y}=\infty$
Thus, $f(x)$ has infinity type of discontinuity at $x=\sqrt{5}$, which is nonremovable type.
$f(x)$ can be redefined at $x=0$ as,
$$
f(x)=\left\{\begin{array}{ll}
\frac{1}{\log _e\left|x^2-4\right|}, & x \neq \pm 2, \pm \sqrt{3}, \pm \sqrt{5} \\
0, & x= \pm 2
\end{array},\right.
$$
which is continuous at $x= \pm 2$.
\end{outline}


\item Test the continuity of $f(x)$ at $x=0$ if
$$
f(x)=\left\{\begin{array}{cl}
(x+1)^{2-\left(\frac{1}{|x|}+\frac{1}{x}\right)}, & x \neq 0 . \\
1, & x=0 .
\end{array}\right.
$$\\\\

\begin{outline}
    Sol. We have $f(0)=0$
L.H.L. at $x=0$,
$$
\begin{aligned}
f\left(0^{-}\right) & =\lim _{x \rightarrow 0^{-}}(x+1)^{2-\left(\frac{1}{|x|}+\frac{1}{x}\right)} \\
& =\lim _{h \rightarrow 0}(0-h+1)^{2-\left(\frac{1}{|0-h|}+\frac{1}{0-h}\right)} \\
& =\lim _{h \rightarrow 0}(1-h)^{2-\left(\frac{1}{h}-\frac{1}{h}\right)} \\
& =\lim _{h \rightarrow 0}(1-h)^2 \\
& =1
\end{aligned}
$$
R.H.L. at $x=0$,
$$
\begin{aligned}
f\left(0^{+}\right) & =\lim _{x \rightarrow 0^{+}}(x+1)^{2-\left(\frac{1}{|x|}+\frac{1}{x}\right)} \\
& =\lim _{h \rightarrow 0}(0+h+1)^{2-\left(\frac{1}{|0+h|}+\frac{1}{0+h}\right)} \\
& =\lim _{h \rightarrow 0}(1+h)^{2-\frac{2}{h}} \\
& =\frac{\lim _{h \rightarrow 0}(1+h)^2}{\lim _{h \rightarrow 0}(1+h)^{\frac{2}{h}}} \\
& =\frac{1}{e^2}
\end{aligned}
$$

Thus, L.H.L. $\neq$ R.H.L., $f(x)$ has jump type of non-removable discontinuity at $x=0$.

\end{outline}


$$
=x-|x||x-1|,-1 \leq x \leq 1 .
$$

Now, using the definition of modulus function we have:
$$
\begin{aligned}
\therefore f(x) & =x-(-x)(1-x) \\
& =2 x-x^2,-1 \leq x<0 \\
f(x) & =x-x(1-x) \\
& =x^2, 0 \leq x \leq 1 .
\end{aligned}
$$

We know that polynomials are continuous everywhere. So, only doubtful point is the turning point $x=0$ of definition.
$$
\begin{aligned}
& \lim _{h \rightarrow 0} f(0+h)=\lim _{h \rightarrow 0}(0+h)^2=0 \\
& \lim _{h \rightarrow 0} f(0-h)=\lim _{h \rightarrow 0} 2(0-h)-(0-h)^2=0 \\
& f(0)=0^2=0
\end{aligned}
$$
so, $f(x)$ is continuous at $x=0$.
Hence, $f(x)$ is continuous in $[-1,1]$.
Hence, the graph of $f(x)$ is continuous in $[-1,1]$ and
$$
\begin{gathered}
f(x)=2 x-x^2 \text { in }-1 \leq x<0 \\
x^2 \text { in } 0 \leq x \leq 1 .
\end{gathered}
$$

\item  If $f(x)=\lim _{x \rightarrow \infty} \frac{\log (x+2)-x^{2 x} \sin x}{x^{2 n}+1}$,
examine the continuity of $f(x)$ at $x=1$.
To examine the continuity at $x=1$, we must derive the definition of $f(x)$ in the intervals $x<1, x>1$ and at $x=1$, i.e., on and around $x=1$.

Now, if $0<x<1$,
$$
\begin{aligned}
f(x) & =\lim _{n \rightarrow \infty} \frac{\log (x+2)-x^{2 n} \sin x}{x^{2 n}+1^2} \\
& =\frac{\log (x+2)-0 \cdot \sin x}{0+1}=\log (x+2)
\end{aligned}
$$
if $x=1, f(x)=\lim _{n \rightarrow \infty} \frac{\log (x+2)-1 \cdot \sin x}{1+1}$
$$
=\frac{\log (x+2)-\sin x}{2}
$$
if $\quad x>1, f(x)=\lim _{n \rightarrow \infty} \frac{\log (x+2)-x^{2 n} \sin x}{x^{2 n}+1}$
$$
=\lim _{n \rightarrow \infty} \frac{\frac{1}{x^{2 n}} \log (x+2)-\sin x}{1+\frac{1}{x^{2 n}}}=-\sin x .
$$

Thus, we have $f(x)=\log (x+2), 0<x<1$
$$
\begin{aligned}
& \frac{1}{2}(\log (x+2)-\sin x), x=1 \\
&-\sin x, x>1 .
\end{aligned}
$$

So $f(x)$ is not continuous at $x=1$.

\item Examine the continuity of $f(x)=\lim _{n \rightarrow \infty} \frac{x}{(2 \sin x)^{2 n}+1}$ for $x \in R$.\\\\

If $|2 \sin x|<1, f(x)=\lim _{n \rightarrow \infty} \frac{x}{(2 \sin x)^{2 n}+1}=\frac{x}{0+1}=x$.
If $|2 \sin x|=1, f(x)=\lim _{n \rightarrow \infty} \frac{x}{(2 \sin x)^{2 n}+1}=\frac{x}{1+1}=\frac{1}{2} x$
If $|2 \sin x|>1, f(x)=\lim _{n \rightarrow \infty} \frac{x}{(2 \sin x)^{2 n}+1}=\frac{x}{\infty+1}=0$.
But $|2 \sin x|<1 \Rightarrow|\sin x|<\frac{1}{2}$
$$
\Rightarrow \quad-\frac{1}{2}<\sin x<\frac{1}{2}
$$
$$
\Rightarrow n \pi-\frac{\pi}{6}<x<n \pi+\frac{\pi}{6}
$$
$|2 \sin x|=1 \quad \Rightarrow \quad|\sin x|=\frac{1}{2}$
$$
\Rightarrow \quad \sin x= \pm \frac{1}{2}
$$
$$
\begin{aligned}
& \Rightarrow \quad x=n \pi+(-1)^m \cdot\left( \pm \frac{\pi}{6}\right)=n \pi \pm \frac{\pi}{6} \\
|2 \sin | x \mid>1 & \Rightarrow|\sin x|>\frac{1}{2} \\
& \Rightarrow \sin x>\frac{1}{2} \quad \text { or } \sin x<-\frac{1}{2} \\
& \Rightarrow n \pi+\frac{\pi}{6}<x<n \pi+\frac{5 \pi}{6} .
\end{aligned}
$$

Thus we have,
$$
\begin{gathered}
f(x)=x, n \pi-\frac{\pi}{6}<x<n \pi+\frac{\pi}{6} \\
\frac{1}{2} x, x=n \pi \pm \frac{\pi}{6} \\
0, n \pi+\frac{\pi}{6}<x<n \pi+\frac{5 \pi}{6} .
\end{gathered}
$$

As polynomial functions are continuous everywhere, only doubtful points are $x=n \pi \pm \frac{\pi}{6}$. Clearly,
$$
\begin{aligned}
& f\left(n \pi-\frac{\pi}{6}+0\right) \neq f\left(n \pi-\frac{\pi}{6}-0\right) \text { because } n \pi-\frac{\pi}{6} \neq 0 \\
& f\left(n \pi+\frac{\pi}{6}+0\right) \neq f\left(n \pi+\frac{\pi}{6}-0\right) \text { because } 0 \neq n \pi+\frac{\pi}{6}
\end{aligned}
$$

ㄱ $f(x)$ is not continuous at $x=n \pi \pm \frac{\pi}{6}$.
$\therefore$ the function $f(x)$ is continuous everywhere in $R$ except the set of points $\left\{x \left\lvert\, x=n \pi \pm \frac{\pi}{6}\right., n \in Z\right\}$.
\end{enumerate}


















\section{Exercises}

1. If $f(x)=\frac{x^2}{a}, 0 \leq x \leq 1$
$a x, 1<x<2$
then assign suitable values to $a$ so that $\lim _{x \rightarrow 1} f(x)$ exists.\\\
2. If $f(x)=\frac{e^{1 / x}}{1+e^{1 / x}}, x \neq 0$
$$
0, x=0
$$
then show that $f(x)$ is discontinuous at $x=0$.\\\
\
3. Examine the continuity of $f(x)=|x-1|+|x+1|$ at $x=1,0$.\\\\
4. By defining $f\left(\frac{\pi}{4}\right)$ suitably, obtain a continuous extension of the function
$$
f(x)=\frac{\sqrt{2} \cos x-1}{\cot x-1}, x \neq \frac{\pi}{4} .
$$\\\\
5. Find $a, b$ and $c$ such that the function
$$
\begin{gathered}
f(x)=\frac{\sin (a+1) x+\sin x}{x}, x<0 \\
c, x=0 \\
\frac{\sqrt{x+b x^2}-\sqrt{x}}{b x^{3 / 2}}, x>0
\end{gathered}
$$
is continuous at $x=0$.\\\\
6. Discuss the continuity of $f(x)$ at $x=0$ where
$$
\begin{aligned}
& f(x)=x\left\{\frac{1}{x(1+x)}+\frac{1}{(1+x)(1+2 x)}\right. \\
& +\frac{1}{(1+2 x)(1+3 x)}+\ldots \text { to } \alpha \\
&
\end{aligned}
$$\\\\
7. Evaluate $\lim _{x \rightarrow 0} \frac{1}{x} \log _e \frac{1+a x}{1-b x}$.
Use this limit to show that the function
$$
\begin{aligned}
f(x)= & \frac{1}{x} \log _e \frac{1+a x}{1-b x}, x \neq 0 \\
a+b, x & =0
\end{aligned}
$$
is continuous at $x=0$.\\\\
8. $f(x)$ is defined as follows:
$$
\begin{aligned}
f(x)= & (\sin x+\cos x)^{\operatorname{cosec} x},-\frac{\pi}{2}<x<0 \\
& , x=0 \\
& \frac{e^{\frac{1}{x}}+e^{\frac{2}{x}}+e^{\frac{3}{|x|}}}{a e^{\frac{2}{x}}+b e^{\frac{3}{|x|}}}, 0<x<\frac{\pi}{2} .
\end{aligned}
$$

If $f(x)$ is continuous at $x=0$, find $a$ and $b$.\\\\
9. If $g(x)=\frac{1-a^x+x a^x \log a}{x^2 \cdot a^x}, x<0$
$$
\frac{(2 a)^x-x \log (2 a)-1}{x^2}, x>0 \quad(\text { where } a>0)
$$
then find $a$ and $g(0)$ so that $g(x)$ is continuous at $x=0$.\\\\
10. Show that the function $f(x)$ defined as follows:
$$
\begin{aligned}
f(x)=2 x+3, x & \in(-3,-2) \\
x+1, x & \in[-2,0] \\
x+2, x & \in(0,1)
\end{aligned}
$$
is discontinuous at $x=0$ and continuous at every other point of the open interval $(-3,1)$. Draw a rough sketch of the function.\\\\
11. Let $f(x)=\frac{x^2}{2}, 0 \leq x \leq 1$
$$
2 x^2-3 x+\frac{3}{2}, 1<x \leq 2 \text {. }
$$

Discuss the continuity of $f(x)$ in $[0,2]$. Give a rough sketch of the function.\\\\
12. Discuss the continuity of $f(x)=\lim _{n \rightarrow \infty} \frac{x^{2 n}-1}{x^{2 n}+1}$ for real $x$.\\\\
13. If $f(x)$ and $g(x)$ are two functions continuous everywhere and
$$
F(x)=\lim _{n \rightarrow \infty} \frac{f(x)+x^{2 n} \cdot g(x)}{1+x^{2 n}}
$$
then prove that $F(x)$ is continuous everywhere except at $x=1,-1$. Find the condition on $f(x), g(x)$, which makes $F(x)$ continuous everywhere.\\\\

14. If $\lim _{x \rightarrow 0} f(x)$ exists where
$$
\begin{aligned}
& f(x)= \frac{\sin (3 p-1) x}{3 x}, x<0 \\
& \frac{\tan (3 p+1) x}{2 x}, x>0 \text { then } p=
\end{aligned}
$$\\\\
15. If $f(x)=a x^2-b, 0 \leq x<1$
$$
\begin{aligned}
& 2, x=1 \\
& x+1,1<x \leq 2
\end{aligned}
$$
is continuous at $x=1$ then $a-b=$\\\\
16. If $f(x)=\frac{1-\sin ^3 x}{3 \cos ^2 x}, x<\frac{\pi}{2}$
$$
\begin{aligned}
& a, x=\frac{\pi}{2} \\
& \frac{b(1-\sin x)}{(\pi-2 x)^2}, x>\frac{\pi}{2}
\end{aligned}
$$
is continuous at $x=\frac{\pi}{2}$ then $a=$ ,$b=$\\\\
17. If $f(x)=x-[x]$ then
$$
\lim _{x \rightarrow 2+0} f(x)=\longrightarrow \lim _{x \rightarrow 2=0} f(x)=
$$\\\\
18. The points at which $f(x)=\cos ^{-1} \frac{x+1}{2}$ has no finite derivative, are\\\\
19.
$$
\text { Let } \begin{aligned}
f(x) & =(x-1)^2 \sin \frac{1}{x-1}-|x|, x \neq 1 \\
& =-1, x=1
\end{aligned}
$$
be a real-valued function. The set of points where $f(x)$ is not differentiable is\\\\
20. Let $f(x)=\frac{x^2+3 x-10}{x-2}, x \neq 2$.
The value $f(2)=$ will make the function $f(x)$ continuous at $x=2$.\\\\
21. If $f(x)$ is continuous in $[0,1]$ and $f\left(\frac{1}{2}\right)=1$ then
$$
\lim _{n \rightarrow-} f\left(\frac{\sqrt{n}}{2 \sqrt{n+1}}\right)=
$$\\\\
22. Let $f(x)=\frac{\sin \left(e^{x-2}-1\right)}{\log (x-1)}, x \neq 2$.
$f(x)$ will be continuous at $x=2$ if $f(2)=$